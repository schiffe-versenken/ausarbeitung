% !TeX document-id = {88ddee8b-8eb6-46e0-9af4-2c10d8b61c04}
% !TeX spellcheck = de-DE
% !TeX encoding = utf8
% !TeX TXS-program:compile = txs:///pdflatex/[--shell-escape]

% This is LLNCS.DEM the demonstration file of
% the LaTeX macro package from Springer-Verlag
\documentclass[a4paper,12pt]{llncs}
%
\usepackage{makeidx}  % allows for indexgeneration
\makeindex

\usepackage[ngerman]{babel}
\usepackage[utf8]{inputenc}      % Code-Page latin 1
\usepackage[T1]{fontenc}
% Nur eine der beiden folgenden Zeilen einbinden!
% siehe Abschnitt Bilder
%\usepackage{graphicx}       % Bilder einbinden, Version fuer normales latex
\usepackage[pdftex]{graphicx}       % Bilder einbinden, Version fuer pdflatex

% mit Hyperrefs
\usepackage[pdftex, plainpages=false,hypertexnames=true,pdfnewwindow=true,backref=true,colorlinks=true,citecolor=blue,linkcolor=black,urlcolor=blue,filecolor=blue]{hyperref}% 
% weitere Packages
\usepackage{ifthen}                 % Zum Auskommentieren von Textteilen
\usepackage{amssymb}                % Mathematische Buchstaben
\usepackage{amsmath}                % Verbesserter Formelsatz
%\usepackage[vlined,boxed]{algorithm2e}
\usepackage{booktabs}               % schönere Tabellen
\usepackage{color}
\usepackage{algpseudocode}			% Für Pseudocode Algorithmen
\usepackage{tcolorbox}
\usepackage{hyperref}
 \hypersetup{urlcolor=black,citecolor=black}

%\setalcapskip{1.5ex} % fuer package algorithm
\usepackage{dsfont}  
%\newtheorem{definition}{Definition}
\usepackage{doc}
\usepackage{mathrsfs}
\usepackage{mathtools}
\usepackage{todonotes}

% Seitenformat ===============================================================
\hoffset=-1.25truecm
\setlength{\topmargin}{0.0cm}
\setlength{\textheight}{23.0cm}
\setlength{\footskip}{1.5cm}
\setlength{\textwidth}{15.4cm}
\setlength{\evensidemargin}{1.5cm}
\setlength{\oddsidemargin}{1.5cm}
\setlength{\parskip}{1ex}
\setlength{\parindent}{0pt}
\setlength{\marginparwidth}{1.4cm}
\setlength{\marginparsep}{1mm}

\pagestyle{plain}

% Makro-Definitionen ==========================================================

% 
\def\myverzeichnis{.}

\numberwithin{equation}{section} 
% Bild -----------------------------------------------------------------------
% #1 Filename;  #2 Label;  #3 Bildunterschrift;  #4 Kurzform
\newcommand{\bild}[4]{
  \begin{figure}[htbp]
    \begin{center}
      \includegraphics{#1}
      \caption[#4]{#3}
      \label{#2}
    \end{center}
  \end{figure}
}

% Bildbreite -----------------------------------------------------------------
% #1 Filename;  #2 Breite;  #3 Label;  #4 Bildunterschrift;  #5 Kurzform
\newcommand{\bildbreite}[5]{
  \begin{figure}[htbp]
    \begin{center}
      \includegraphics[width=#2]{#1}
      \caption[#5]{#4}
      \label{#3}
    \end{center}
  \end{figure}
}

\newtheorem{satz}{Satz}
\newtheorem{korollar}{Korollar}

\DeclareMathOperator{\hit}{hit}
\DeclareMathOperator{\shot_count}{shot\_count}
\DeclareMathOperator{\strat}{strat}
\DeclareMathOperator{\strategies}{strategies}
\DeclareMathOperator{\sunk}{sunk}


% ============================================================================
\begin{document}

% =========== Das war der Vorspann, jetzt geht's los! ========================

% ============================================================================
% =============  AB HIER DARF UND SOLL GETIPPT WERDEN ========================
% ============================================================================
\author{Viel Schreiber}
\index{Viel Schreiber}

% Das Institut wird fuer den Betreuer missbraucht ...
\institute{{\bf Betreuer:} Dipl.-Inf. Carl Coder}
\authorrunning{Viel Schreiber}
\title{Meine Seminarausarbeitung}

\maketitle

%\todo[inline]{Version \gitAbbrevHash, Branch \gitBranch, \gitCommitterIsoDate \gitDirty}

\thispagestyle{empty}

\begin{abstract}
Ein schöner Abstract. Das ist einfach die Kurzzusammenfassung.
\end{abstract}

% Einleitung -----------------------------------------------------------------
\section{Einleitung}

\subsection{Spielbeschreibung}
\nocite{WB95} % Mindestens eine Zitierung ist nötig, damit BibTex funktioniert
Diese Ausarbeitung beschäftigt sich mit der Umsetzung von Schiffe-Versenken in beliebig viele Dimensionen.
Die generelle Funktionsweise von dem normalen Schiffe-Versenken bleibt erhalten, muss jedoch um einige Dinge erweitert werden, um auch in höheren Dimensionen gut spielbar zu sein.

Zu Beginn platziert der Gegner eine beliebige Anzahl an Schiffen auf seinem Spielfeld. Es wird davon ausgegangen, dass die Schiffspositionen rein zufällig ausgewählt wurden, d.h. dass zu Spielbeginn eine Flotte aus der Menge aller möglichen Flotten ausgewählt wird. Jede mögliche Flotte hat hierbei die gleiche Wahrscheinlichkeit, ausgewählt zu werden.
Natürlich sind die Schiffspositionen dem Spieler nicht bekannt.
Die Gesamtheit aller gewählten Schiffspositionen wird auch die Flotte genannt.
Nun kann der Spieler anfangen, auf bestimmte Positionen auf dem Spielfeld, auch Zellen genannt, zu schießen.
Da Schiffe überlappen können, können mit jedem Schuss mehrere Schiffe, also Teilmengen der Flotte getroffen werden.
Nach jedem Schuss erfährt der Spieler von seinem Gegner, welche Teilmenge der Flotte er getroffenen hat.
Falls keine Schiffe getroffen wurden ist diese Menge natürlich leer.
Ein Schiff gilt als versenkt, sobald es getroffen wurde.
Sobald alle Schiffe versenkt wurden, ist das Spiel beendet.
Das Ziel des Spielers ist es, mit möglichst wenig Schüssen alle Schiffe zu versenken.

\subsection{Inhalt der Ausarbeitung}

Der erste Teil dieser Ausarbeitung beschäftigt sich mit der Bewertung einer gegebenen Strategie, d.h. mit der Berechnung der erwarteten Anzahl an Schüssen, die mit einer Strategie zum versenken der platzierten Flotte gebraucht werden.
Da diese Berechnung sehr aufwändig ist, wird ein Monte-Carlo Algorithmus zur näherungsweisen Berechnung beschrieben, implementiert und ausgewertet.

Im zweiten Teil werden unter den zuvor etablierten Rahmenbedingungen verschiedene Strategien vorgestellt. Dabei können sich diese von denen in der Bachelorarbeit vorgestellten unterschieden, da es sich um andere Rahmenbedingungen handelt. Zu den Strategien zählen dünne und volle Gitter, klassisches und Quasi Monte Carlo und die optimale Maximum Strategie.

Dritter Teil (Implementierung + Vergleich) TODO 

\section{Formalisierung des Spielprinzips}

\begin{definition}
Sei
\[
C_{all} \coloneqq \{1, \dots, N\}^d
\]
die Menge aller Zellen des Spielfeldes mit jeweils $N$ Zellen in $d$ Dimensionen.
\end{definition}

\begin{satz}
Dann gilt:
\[
|C_{all}|=N^d
\]
\end{satz}

\begin{definition}
Seien $c, c' \in C_{all}$ Zellen.
Dann ist
\[
c \leq c' \Leftrightarrow \forall i \in \{1, \dots, d\} \colon c_{i} \leq c'_{i} 
\]
\end{definition}

\begin{definition}
Sei $c_{min} \in C_{all}$ und $c_{max} \in C_{all}$ mit $c_{min} \leq c_{max}$.
\[
l=(c_{min}, c_{max})
\]
eine mögliche Schiffsposition (location), welche mithilfe einer minimalen Ecke $c_{min}$ und einer maximalen Ecke $c_{max}$ bestimmt wird.
\end{definition}

\begin{definition}
Sei 
\[
L_{all} \coloneqq
\{
(i, j) \in C_{all} \times C_{all}
\mid
i \leq j
\}
\] die Menge aller möglichen Schiffspositionen.
\end{definition}

\begin{satz}
Dann gilt:
\[
|L_{all}|=\left(\frac{(N+1) N}{2}\right)^d
\]
\end{satz}

\begin{proof}
Da das Spielfeld in jeder Dimension die Größe $N$ hat, gilt in jeder Dimension $d$ für gültige Start- und Endkoordinaten eines Schiffes $c_d$ und $c'_d$ mit $c_d \leq c'_d$ folgendes:
\begin{align}
\begin{split}
&|\{(c_d, c'_d) \in N \times N \mid c_d \leq c'_d\}|\\
&=|\{(i, j) \in N \times N \mid i \leq j\}|\\
&=\sum_{i=1}^N N - i + 1\\
&=N + \sum_{i=1}^{N-1} i\\
&=\sum_{i=1}^{N} i\\
&= \frac{(N + 1) N}{2}
\nonumber
\end{split}
\end{align}
Daher gibt es genau
\[
\left(\frac{(N+1) N}{2}\right)^d
\]
gültige Schiffspositionen, da für jede Schiffsposition für jede Dimension gültige Start- und Endkoordinaten gewählt werden müssen, also genau $d$ gültige Start- und Endkoordinaten.
\end{proof}

\begin{definition}
Sei 
\[
K \coloneqq \{1, \dots, |L_{all}|\}
\]
die Menge aller möglichen Flottengrößen.
\end{definition}

\begin{definition}
Sei $k \in K$ die Größe der Flotte.
Dann ist
\[
F_{all,k} \coloneqq\{F \in \mathcal{P}(L_{all}) \mid |F| = k\}
\]
die Menge aller Flotten der Größe $k$.
\end{definition}

\begin{satz}
Sei $k \in K$ die Größe der Flotte.
Dann gilt:
\[
|F_{all,k}|=\binom{|L_{all}|}{k}
\]
\end{satz}

\begin{definition}
Dann ist
\[
F_{all} \coloneqq \bigcup_{k=1}^{|L_{all}|} F_{all,k} = \mathcal{P}(L_{all}) \setminus \{\emptyset\}
\]
\end{definition}

\begin{satz}
Dann gilt:
\[
|F_{all}|=2^{|L_{all}|} - 1
\]
\end{satz}

\begin{proof}
\begin{align}
\begin{split}
&|F_{all}|=\sum_{k=1}^{|L_{all}|} |F_{all,k}|\\
=&\sum_{k=1}^{|L_{all}|} \binom{|L_{all}|}{k} \\
=&\left( \sum_{k=0}^{|L_{all}|} \binom{|L_{all}|}{k} \right) - 1 \\
=&2^{|L_{all}|} - 1
\end{split}
\end{align}
\qed
\end{proof}


\begin{definition}
Sei $F \in F_{all}$ und $c \in C_{all}$.
Dann ist 
\begin{align}
&\hit:F_{all} \times C_{all} \rightarrow \mathcal{P}(L_{all}) \quad mit \nonumber\\
&\hit(F, c)\mapsto \{(c_{min}, c_{max}) \in F \mid c_{min} \leq c \leq c_{max}\} \nonumber
\end{align}
die Treffer-Funktion, welche angibt, welche Schiffe bei einem Schuss auf Zelle $c$ getroffen wurden, falls $F$ die vom Gegner platzierte Flotte ist. In anderen Worten, die Menge der Schiffe aus Flotte $F$, die die Zelle $c$ belegen.
\end{definition}

\subsection{Schuss-Strategien}

\begin{definition}
Eine Funktion der Form
\begin{align}
&\strat:\mathcal{P}(C_{all}) \rightarrow C_{all} \nonumber
\end{align}
wird Strategiefunktion genannt. Diese wählt die nächste Zelle aus, auf die geschossen werden soll, abhängig davon, auf welche Zellen bereits geschossen wurden.
\end{definition}

\begin{definition}
Sei
\[
\strategies  \coloneqq \{ \strat:P(C_{all}) \rightarrow C_{all} \}
\]
die Menge an allen Strategiefunktionen.
\end{definition}

\begin{definition}
Sei $F\in F_{all}$ die gewählte Flotte und $S \in P(C_{all})$ die Menge an bereits beschossenen Zellen.
Dann ist:
\[
\sunk(F, S) \Leftrightarrow \bigcup_{c \in S} \hit(F, c) = F
\]
wahr, falls alle Schiffe aus der Flotte $F$ versenkt wurden.
\end{definition}


\begin{definition}
Sei $F\in F_{all}$ die gewählte Flotte und $S$ die Menge an bereits beschossenen Zellen.
Sei außerdem $\strat \in \strategies$ die verwendete Schuss-Strategie.
Dann ist
\begin{align}
&\shot_count(F, S, \strat)=
& \begin{cases} 
  	0& ,\sunk(F, S) \\
      \shot_count(F, S \cup \strat(S), \strat) + 1 & ,sonst
   \end{cases}
\nonumber
\end{align}
die Anzahl an Schüssen, die benötigt werden, um alle Schiffe der Flotte $F$ mit der Schuss-Strategiefunktion $\strat$ zu versenken, falls bereits auf die Zellen der Menge $S$ geschossen wurde.

Außerdem ist dann
\begin{align}
&\shot_count(F, \strat)=\shot_count(F, \emptyset, \strat)
\nonumber
\end{align}
die Anzahl an Schüssen, die benötigt werden, um alle Schiffe der Flotte $F$ mit der Schuss-Strategiefunktion $\strat$ zu versenken.
\end{definition}

\subsection{Zufallsvariablen}

\begin{definition}
Sei $\Omega \subseteq F_{all}$ eine Menge an Flotten.
\end{definition}

\begin{definition}
Sei $A \subseteq \Omega$ das Ereignis, dass bei einem Auswählen einer Flotte aus $\Omega$, die Flotte ebenfalls in der Menge $A$ liegt.
Dann ist
\[
P(A) = \frac{|A|}{|\Omega|}
\]
die Wahrscheinlichkeit, dass eine Flotte aus der Menge $A$ ausgewählt wird.
\end{definition}

Dann ist $(\Omega, P)$ ein endlicher Laplacescher W-Raum und P eine Gleichverteilung.

\begin{definition}
Sei $\strat \in \strategies$ die verwendete Schuss-Strategie.
Sei $\omega$ die ausgewählte Flotte aus $\Omega$.
Dann ist
\begin{align}
\mathbf{T}_{\strat}(\omega) \coloneqq \shot_count(\omega, \strat)
\nonumber
\end{align}
eine Zufallsvariable, die die Anzahl an Schüssen zum versenken einer Flotte $\omega$ bezeichnet.
\end{definition}

\begin{satz}
Sei $\strat \in \strategies$ die verwendete Schuss-Strategie.
Dann ist
\begin{align}
E_\Omega(\mathbf{T}_{\strat})=\frac{1}{|\Omega|} \sum_{\omega \in \Omega} \shot_count(\omega, \strat)
\nonumber
\end{align}
der Erwartungswert von $\mathbf{T}_{\strat}$ in der Menge $\Omega$.
\end{satz}

\begin{proof}
\begin{align}
\begin{split}
E_\Omega(\mathbf{T}_{\strat})&=\sum_{\omega \in \Omega} P(\omega) \shot_count(\omega, \strat)\\
&=\frac{1}{|\Omega|} \sum_{\omega \in \Omega} \shot_count(\omega, \strat)
\nonumber
\end{split}
\end{align}
\qed
\end{proof}

\begin{satz}
Sei $\strat \in \strategies$ die verwendete Schuss-Strategie.
Dann ist
\begin{align}
E_{F_{all}}(\mathbf{T}_{\strat})=\frac{1}{|F_{all}|} \sum_{k=1}^{|L_{all}|}\sum_{F\in F_{all,k}} \shot_count(F, \strat)
\nonumber
\end{align}
der Erwartungswert von $\mathbf{T}_{\strat}$ in der Menge $F_{all}$.
\end{satz}

\begin{proof}
\begin{align}
\begin{split}
&E_{F_{all}}(\mathbf{T}_{\strat})\\
=&\frac{1}{|F_{all}|} \sum_{F\in F_{all}} \shot_count(F, \strat)\\
=&\frac{1}{|F_{all}|} \sum_{k=1}^{|L_{all}|}\sum_{F\in F_{all,k}} \shot_count(F, \strat)
\nonumber
\end{split}
\end{align}
\qed
\end{proof}

\begin{definition}
Sei $k \in K$ die Flottengröße.
Dann ist
\[
g_k=\frac{|F_{all,k}|}{|F_{all}|}=\frac{\binom{|L_{all}|}{k}}{2^{|L_{all}|} - 1}
\]
der Anteil an Flotten der Größe $k$ an der Menge aller Flotten. Dieser Faktor wird später als Gewichtungsfaktor für den Erwartungswert der Flottengröße $k$ verwendet.
\end{definition}

\begin{satz}
Sei $\strat \in \strategies$ die verwendete Schuss-Strategie.
Dann ist
\begin{align}
E_{F_{all}}(\mathbf{T}_{\strat})=\sum_{k=1}^{|L_{all}|} g_k E_{F_{all,k}}(\mathbf{T}_{\strat})
\nonumber
\end{align}
der Erwartungswert von $\mathbf{T}_{\strat}$ in der Menge $F_{all}$.
\end{satz}

\begin{proof}
\begin{align}
\begin{split}
&E_{F_{all}}(\mathbf{T}_{\strat})\\
=&\frac{1}{|F_{all}|} \sum_{k=1}^{|L_{all}|}\sum_{F\in F_{all,k}} \shot_count(F, \strat)\\
=&\sum_{k=1}^{|L_{all}|} \frac{1}{|F_{all}|} \sum_{F\in F_{all,k}} \shot_count(F, \strat)\\
=&\sum_{k=1}^{|L_{all}|} \frac{1}{|F_{all}|} |F_{all,k}| \sum_{F\in F_{all,k}} \frac{1}{|F_{all,k}|} \shot_count(F, \strat)\\
=&\sum_{k=1}^{|L_{all}|} \frac{|F_{all,k}|}{|F_{all}|} E_{F_{all,k}}(\mathbf{T}_{\strat})\\
=&\sum_{k=1}^{|L_{all}|} g_k E_{F_{all,k}}(\mathbf{T}_{\strat})
\nonumber
\end{split}
\end{align}
\qed
\end{proof}

\section{Berechnung des Erwartungswertes mithilfe eines MC-Algorithmus}

Um zu bewerten, wie gut eine Strategiefunktion $\strat$ abschneidet, kann ein Monte-Carlo Algorithmus implementiert werden, der für jede Flottengröße $k \in K$ mit einer Stichprobe den Wert von $E_{F_{all,k}}(\mathbf{T}_{\strat})$ abschätzt und am Ende alle diese Werte zum gesamtem Erwartungswert $E_{F_{all}}(\mathbf{T}_{\strat})$ kombiniert.

Um den Wert von $\shot_count(F, \strat)$ für eine gegebene Flotte $F$ zu berechnen, kann folgende, (naive) Funktion verwendet werden:

\begin{tcolorbox}
	\begin{algorithmic}[H]
		\Function{shot\_count}{$F, \strat$}{:}
		\State $S=\emptyset$;
		\State $\shot_count=0$;
		\While{$\neg \sunk(F, S)$}
		\State $next\_target=\strat(S)$;
		\State $S=S \cup \hit(F, next\_target)$;
		\State $\shot_count=\shot_count+1$;
		\EndWhile
		\State\Return $\shot_count$;
		\EndFunction
	\end{algorithmic}
\end{tcolorbox}

Um damit dann einen kompletten Algorithmus implementieren zu können, muss jedoch erst eine Abbruchbedingung formuliert werden, die bestimmt, wann für eine Flottengröße $k \in K$ der approximierte Wert von $E_{F_{all,k}}(\mathbf{T}_{\strat})$ gut genug ist.

Dafür ist es nützlich, die Erwartungswertänderung wie in dem folgenden Satz darzustellen:

\begin{satz}
Sei $G_k^{(n)} \subseteq F_{all,k}$ eine Teilmenge der Flotten mit Größe $k$ mit $|G_k^{(n)}|=n$.
Sei außerdem $F_k^{(n+1)} \in (F_{all,k} \setminus G_k^{(n)})$ eine weitere, noch nicht ausgewertete Flotte.
Dann ist $G_k^{(n+1)}=G_k^{(n)} \cup F_k^{(n+1)}$ und für den Erwartungswert gilt:
\begin{align}
E_{G_k^{(n+1)}}(\mathbf{T}_{\strat})=E_{G_k^{(n)}}(\mathbf{T}_{\strat}) + \delta\\
 \delta=\frac{\mathbf{T}_{\strat}(F_k^{(n+1)}) - E_{G_k^{(n)}}(\mathbf{T}_{\strat})}{n+1}
\nonumber
\end{align}
\end{satz}

\begin{proof}
Sei
\begin{align}
\begin{split}
s=\sum_{F \in G_k^{(n)}} \mathbf{T}_{\strat}(F),\;\; E_{G_k^{(n)}}(\mathbf{T}_{\strat}) =\frac{s}{n}
\end{split}
\end{align}
Dann ist
\begin{align}
\begin{split}
&E_{G_k^{(n+1)}}(\mathbf{T}_{\strat})\\
=&\frac{s+ \mathbf{T}_{\strat}(F_k^{(n+1)}) }{n + 1}\\
=&\frac{s}{n + 1} + \frac{\mathbf{T}_{\strat}(F_k^{(n+1)}) }{n + 1}\\
=&\frac{s}{n} * \frac{n}{n + 1} + \frac{\mathbf{T}_{\strat}(F_k^{(n+1)}) }{n + 1}\\
=&E_{G_k^{(n)}}(\mathbf{T}_{\strat}) * (1 - \frac{1}{n + 1}) + \frac{\mathbf{T}_{\strat}(F_k^{(n+1)}) }{n + 1}\\
=&E_{G_k^{(n)}}(\mathbf{T}_{\strat}) - \frac{E_{G_k^{(n)}}(\mathbf{T}_{\strat})}{n + 1}) + \frac{\mathbf{T}_{\strat}(F_k^{(n+1)}) }{n + 1}\\
=&E_{G_k^{(n)}}(\mathbf{T}_{\strat}) + \frac{E_{G_k^{(n)}}(\mathbf{T}_{\strat}) - \mathbf{T}_{\strat}(F_k^{(n+1)}) }{n + 1}\\
\end{split}
\end{align}
\qed
\end{proof}

Damit kann dann eine Abbruchbedingung formuliert werden, die angibt, wie groß die Flottenanzahl $n$ sein muss, damit Veränderung des Erwartungswertes $\delta$ für eine zusätzlich ausgewertete Flotte $F_k^{(n+1)}$ einen Grenzwert $\delta_{min}$ immer unterschreitet:

\begin{satz}
Sei $\delta_{min} > 0$.
Dann gilt:

\begin{align}
n = \lceil \frac{|C_{all}| - 1}{\delta_{min}} \rceil - 1 \Rightarrow |\delta| \leq \delta_{min}
\nonumber
\end{align}
\end{satz}

\begin{proof}
\begin{align}
|\delta|=\frac{|\mathbf{T}_{\strat}(F_k^{(n+1)}) - E_{G_k^{(n)}}(\mathbf{T}_{\strat})|}{n+1} \leq \delta_{min}\\
\Leftrightarrow \frac{|\mathbf{T}_{\strat}(F_k^{(n+1)}) - E_{G_k^{(n)}}(\mathbf{T}_{\strat})|}{\delta_{min}} \leq (n+1)
\nonumber
\end{align}
Da $\forall F \in F_{all} \colon \mathbf{T}_{\strat}(F) \leq |C_{all}|$ und $\forall G \subseteq F_{all} \colon E_{G}(\mathbf{T}_{\strat}) \geq 1$ gilt:
\begin{align}
\begin{split}
\left( (n+1) \geq \frac{|C_{all}| - 1}{\delta_{min}} \Rightarrow |\delta| \leq \delta_{min} \right)\\
\Rightarrow \left(n = \lceil \frac{|C_{all}| - 1}{\delta_{min}} \rceil - 1 \Rightarrow |\delta| \leq \delta_{min} \right)
\end{split}
\end{align}
\qed
\end{proof}

Diese Abschätzung ist sehr großzugüg gewählt. Es wird sehr oft passieren, dass bereits für kleinere Werte von $n$ der Grenzwert $\delta_{min}$ unterschritten wird, jedoch kann mit dieser größeren Wahl von $n$ garantiert werden, dass der Grenzwert $\delta_{min}$ in jedem zusätzlichen Schritt unterschritten wird. Daher muss nicht nach jedem Schritt eine Abbruchbedingung überprüft werden, sondern nur zu Beginn die Anzahl der Schritte $n$ einmalig berechnet werden.

\begin{tcolorbox}
	\begin{algorithmic}[H]
		\Function{simulate\_rounds}{$k, \strat, \delta_{min}$}{:}
		\State $n =\lceil \frac{|C_{all}| - 1}{\delta_{min}} \rceil - 1$;
		\State $sum=0$;
		\For{$i=1,\dots,n$}
		\State $sum += shot\_count(generate\_random\_fleet(k), \strat)$;
		\EndFor
		\State\Return $\frac{sum}{n}$;
		\EndFunction
	\end{algorithmic}
\end{tcolorbox}


Abschliessend kann dann die Funktion implementiert werden, die den Wert von $E_{F_{all}}(\mathbf{T}_{\strat})$ für Flotten beliebiger Größe berechnet. Dafür müssen die berechneten Werte von $E_{F_{all,k}}(\mathbf{T}_{\strat})$ für jedes $k$ gewichtet aufsummiert werden. Nebenbei werden auch die Erwartungswerte für jedes $k$ gespeichert, damit diese später auch untersucht werden können.

\begin{tcolorbox}
\begin{algorithmic}[H]
	\Function{calculate\_expected\_shot\_count}{strat, $\epsilon, \eta$}{:}
	\State $expected\_shot\_count=[]$;
	\State $total\_expected\_shot\_count=0$;
	\For{$k=1$ to $|L_{all}|$}
	\State $g_k=\frac{|F_{all,k}|}{|F_{all}|}$;
	\State $\delta_k=g_k * \delta$;
	\State $E=simulate\_rounds(k, strat, \delta_k)$;
	\State $expected\_shot\_count[k] = E$;
	\State $total\_expected\_shot\_count+=g_k * E$;
	\EndFor
	\EndFunction
\end{algorithmic}
\end{tcolorbox}


\section{Strategien}

Der folgende Abschnitt befasst sich mit Schussstrategien. In \cite{M13} wurden diskrepanzminimierende Diskretisierungsverfahren erstmals als potentielle Schussstrategien eingeführt. Dabei wurden die von diesen Verfahren genutzten Stützstellen meist als Schüsse interpretiert. Zu Beginn wird noch einmal die optimale Strategie aus \cite{M13} aufgegriffen, da sich die Rahmenbedingungen dieser Ausarbeitung doch von \cite{M13} unterscheiden, um diese in der hier eingeführten Notation darzustellen. Anschließend werden noch andere Strategien, wie dünne und volle Gitter, klassisches und Quasi Monte-Carlo betrachtet. 

Sei für alle folgenden Strategien $F\in F_{all}$ die aktuell vorliegende unbekannte Flotte.

\subsection{Maximum Strategie}

Diese in \cite{M13} Abbildung 3.15 vorgestellte Strategie berechnet zunächst eine Anfangsverteilung. Dabei wird für jeden Punkt im betrachteten $d$-dimensionalen Raum festgestellt wie viele Schiffe diesen Punkt enthalten. Anschließend wird dem Punkt diese Anzahl zugewiesen. Von diesen Punkten wird dann das Maximum bezüglich dieser Anzahl bestimmt. Das ist dann die Zelle auf die als nächstes geschossen wird.

In \cite{M13} wurden alle möglichen Schiffspositionen auf einmal betrachtet und von diesen Verteilungen die Maxima berechnet. Allerdings betrachten wir in dieser Ausarbeitung nur eine unbekannte Flotte $F$ in vielen Experimenten und versuchen dann durch eine Monte-Carlo Zusammenführung eine Aussage über alle Flotten zu erreichen. Das heißt in diesem Fall wird das Maximum an einer anderen Stelle berechnet als noch in \cite{M13}. 

Der Algorithmus für die Maximum bzw. optimale Strategie sieht dann folgendermaßen aus:

\begin{tcolorbox}
	\begin{algorithmic}
		\Function{max\_strat}{$S$}{$:cell$}
		\State $cell = null$;
		\State $max = 0$;
		\State $temp = 0$;
		\For{\textbf{each } $c\in C_{all}$}
		\State $temp=\left|\hit(F,c)\right|$;
		\If{$temp>max$}
		\State $max=temp$;
		\State $cell = c$;
		\EndIf
		\EndFor
		\State \Return $cell$;
		\EndFunction
	\end{algorithmic}
\end{tcolorbox}

Wie man sieht bewegen wir uns hier bereits in einer Komplexität von $O(\left|C_{all}\right|\cdot O(hit))$. %O(hit) aktuell noch unbekannt, daher unformelle Notation.

\subsection{Gitter Strategien}

In Abschnitt 2.2.1 von \cite{M13} werden die hierarchischen Stützstelleninkremente $H_{\underline{m}}$ für Gitter vorgestellt, welche sich von den in \cite{M13} Abbildung 2.2 dargestellten Hütchenfunktionen ableiten. Diese Inkremente bzw. ihre Kombinationen werden sowohl für dünne als auch volle Gitter genutzt. Für unsere Zwecke genügt, dass sich einerseits die Maschenweite mit steigendem Level in eine beliebige Dimension halbiert (Vgl. \cite{P10} Figure 2.5, schön zu sehen, dass Stützstellen höherer Level zwischen den Linien, die durch ihre Vorgänger entstanden, liegen) und andererseits durch Kombinationen von Inkrementen Gitter erzeugen lassen. Unsere Schussfolge ergibt sich dann aus den abgearbeiteten Inkrementen bzw. deren Stützstellen. 

Noch zu beachten ist, dass es bei Gittern so genannte Level gibt, die die Maschenweite und damit die Genauigkeit angeben. Normalerweise legt man ein Level fest und erzeugt sich dann aus Inkrementen ein Gitter eben dieses Levels. In unseren Fall arbeiten wir uns durch die Inkremente aufsteigend durch um immer genauere Gitter zu erzeugen bei steigendem Level, bis wir alle Schiffe der Flotte getroffen haben.

\subsubsection{Volle Gitter Strategie}

In \cite{M13} Definition 2.14 wird beschrieben, wie sich volle Gitter aus den Inkrementen zusammensetzen. Dabei ist lediglich zu beachten, dass die Indeces eines Inkrements $H_{\underline{m}}$ nur kleiner dem Level $n$ sein müssen.  Zur Wiederholung:

\begin{definition}
	Seien $d,n\in\mathbb{N}$ und $d,n>1$. Ein $d$-dimensionales volles Gitter $G$ mit Level $n$ wird durch Kombination der Inkremente $H_{\underline{m}}$ mit $\underline{m}<(n,\dots,n)$ erzeugt:
	\begin{equation}
		G=\biguplus_{m_1=1}^n\biguplus_{m_2=1}^n\dots \biguplus_{m_d=1}^n H_{\underline{m}}
	\end{equation}
\end{definition}

\subsubsection{Dünne Gitter Strategie}

In \cite{M13} Definition 2.15 wird beschrieben, wie sich dünne Gitter aus den Inkrementen zusammensetzen. Zur Wiederholung:

\begin{definition}
	Seien $d,n\in\mathbb{N}$ und $d,n>1$. Ein $d$-dimensionales dünnes Gitter $G$ mit Level $n$ wird durch Kombination der Inkremente $H_{\underline{m}}$ mit $|\underline{m}|_1\leq n+d-1$ erzeugt:
	\begin{equation}
	G=\biguplus_{|\underline{m}|_1\leq n+d-1}^n H_{\underline{m}}
	\end{equation}
\end{definition}

Wobei für die verwendete Norm gilt (Vgl. \cite{P10} Abschnitt 2.1 Basics):

\begin{definition}
Die  $|\cdot|_1$-Norm ist definiert als:
	\begin{equation}
	|\underline{m}|_1:=\sum_{j=1}^d m_j
	\end{equation}
\end{definition}


\subsection{Monte-Carlo Strategie}

Ähnlich wie Strategien aus der Diskretisierung durch dünne und volle Gitter gewonnen wurden, kann eine Strategie aus der Monte-Carlo Diskretisierung abgeleitet werden. Dabei stellt die Wahl der Stützstellen, die bei Gittern noch nach Gesetzmäßigkeiten ausgewählt wurden, effektiv die Strategie dar, so dass diese Punkte den Schüssen entsprechen. Da bei Monte-Carlo diese Stützstellen zufällig gewählt werden, tun wir dies auch hier für unsere (erste) Monte-Carlo Strategie. 

\subsection{Quasi-Monte-Carlo Strategie}


% Literaturverzeichnis ------------------------------------------------
\newpage
\bibliographystyle{alphadinLinkLocal}
%\bibliography{literatur} 

\begin{thebibliography}{WB95}
	\providecommand{\url}[1]{\texttt{#1}}
	\expandafter\ifx\csname urlstyle\endcsname\relax
	\providecommand{\doi}[1]{doi: #1}\else
	\providecommand{\doi}{doi: \begingroup \urlstyle{rm}\Url}\fi
	
	\bibitem[WB95]{WB95}
	\textsc{Welch}, G. ; \textsc{Bishop}, G.:
	\newblock An Introduction to the Kalman Filter  / UNC-CH Computer Science.
	\newblock \,Version:\,1995.
	\newblock  \url{http://www.cs.unc.edu/~welch/media/pdf/kalman_intro.pdf}
	(95-041). --
	\newblock Technical Report. --
	\newblock Online--Ressource
	
	\bibitem[M13]{M13}
	\textsc{Mehlbeer}, F.:
	\newblock \textit{Hierarchische Methoden am Beispiel von Schiffe versenken},
	\newblock Bachelorarbeit Nr.25,
	\newblock Universität Stuttgart,
	\newblock 2013
	
	\bibitem[P10]{P10}
	\textsc{Pflüger}, D.:
	\newblock \textit{Spatially Adaptive Sparse Grids for High-Dimensional Problems},
	\newblock Dissertation,,
	\newblock Technische Universität München,
	\newblock 2010
	
\end{thebibliography}


%\iffalse
\end{document}
%\fi
