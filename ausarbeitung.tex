% !TeX document-id = {88ddee8b-8eb6-46e0-9af4-2c10d8b61c04}
% !TeX spellcheck = de-DE
% !TeX encoding = utf8
% !TeX TXS-program:compile = txs:///pdflatex/[--shell-escape]

% This is LLNCS.DEM the demonstration file of
% the LaTeX macro package from Springer-Verlag
\documentclass[a4paper,12pt]{llncs}
%
\usepackage{makeidx}  % allows for indexgeneration
\makeindex

\usepackage[ngerman]{babel}
\usepackage[utf8]{inputenc}      % Code-Page latin 1
\usepackage[T1]{fontenc}
% Nur eine der beiden folgenden Zeilen einbinden!
% siehe Abschnitt Bilder
\usepackage{graphicx}       % Bilder einbinden, Version fuer normales latex
%\usepackage[pdftex]{graphicx}       % Bilder einbinden, Version fuer pdflatex

% mit Hyperrefs
\usepackage[pdftex, plainpages=false,hypertexnames=true,pdfnewwindow=true,backref=true,colorlinks=true,citecolor=blue,linkcolor=black,urlcolor=blue,filecolor=blue]{hyperref}% 
% weitere Packages
\usepackage{ifthen}                 % Zum Auskommentieren von Textteilen
\usepackage{amssymb}                % Mathematische Buchstaben
\usepackage{amsmath}                % Verbesserter Formelsatz
%\usepackage[vlined,boxed]{algorithm2e}
\usepackage{booktabs}               % schönere Tabellen
\usepackage{color}
\usepackage{algpseudocode}			% Für Pseudocode Algorithmen
\usepackage{tcolorbox}
\usepackage{hyperref}
 \hypersetup{urlcolor=black,citecolor=black}

%\setalcapskip{1.5ex} % fuer package algorithm
\usepackage{dsfont}  
%\newtheorem{definition}{Definition}
\usepackage{doc}
\usepackage{mathrsfs}
\usepackage{mathtools}
\usepackage{todonotes}
\usepackage{subfigure}

% Seitenformat ===============================================================
\hoffset=-1.25truecm
\setlength{\topmargin}{0.0cm}
\setlength{\textheight}{23.0cm}
\setlength{\footskip}{1.5cm}
\setlength{\textwidth}{15.4cm}
\setlength{\evensidemargin}{1.5cm}
\setlength{\oddsidemargin}{1.5cm}
\setlength{\parskip}{1ex}
\setlength{\parindent}{0pt}
\setlength{\marginparwidth}{1.4cm}
\setlength{\marginparsep}{1mm}

\pagestyle{plain}

% Makro-Definitionen ==========================================================

% 
\def\myverzeichnis{.}

\numberwithin{equation}{section} 
% Bild -----------------------------------------------------------------------
% #1 Filename;  #2 Label;  #3 Bildunterschrift;  #4 Kurzform
\newcommand{\bild}[4]{
  \begin{figure}[htbp]
    \begin{center}
      \includegraphics{#1}
      \caption[#4]{#3}
      \label{#2}
    \end{center}
  \end{figure}
}

% Bildbreite -----------------------------------------------------------------
% #1 Filename;  #2 Breite;  #3 Label;  #4 Bildunterschrift;  #5 Kurzform
\newcommand{\bildbreite}[5]{
  \begin{figure}[htbp]
    \begin{center}
      \includegraphics[width=#2]{#1}
      \caption[#5]{#4}
      \label{#3}
    \end{center}
  \end{figure}
}

\newtheorem{satz}{Satz}
\newtheorem{korollar}{Korollar}

\DeclareMathOperator{\hit}{hit}
\DeclareMathOperator{\shot_count}{shot\_count}
\DeclareMathOperator{\strat}{strat}
\DeclareMathOperator{\strategies}{strategies}
\DeclareMathOperator{\sunk}{sunk}


% ============================================================================
\begin{document}

% =========== Das war der Vorspann, jetzt geht's los! ========================

% ============================================================================
% =============  AB HIER DARF UND SOLL GETIPPT WERDEN ========================
% ============================================================================
\author{Viel Schreiber}
\index{Viel Schreiber}

% Das Institut wird fuer den Betreuer missbraucht ...
\institute{{\bf Betreuer:} Dipl.-Inf. Carl Coder}
\authorrunning{Viel Schreiber}
\title{Meine Seminarausarbeitung}

\maketitle

%\todo[inline]{Version \gitAbbrevHash, Branch \gitBranch, \gitCommitterIsoDate \gitDirty}

\thispagestyle{empty}

\begin{abstract}
Ein schöner Abstract. Das ist einfach die Kurzzusammenfassung.
\end{abstract}

% Einleitung -----------------------------------------------------------------
\section{Einleitung}

\subsection{Spielbeschreibung}
\nocite{WB95} % Mindestens eine Zitierung ist nötig, damit BibTex funktioniert
Diese Ausarbeitung beschäftigt sich mit der Umsetzung von Schiffe-Versenken in beliebig viele Dimensionen.
Die generelle Funktionsweise von dem normalen Schiffe-Versenken bleibt erhalten, muss jedoch um einige Dinge erweitert werden, um auch in höheren Dimensionen gut spielbar zu sein.

Zu Beginn platziert der Gegner eine beliebige Anzahl an Schiffen auf seinem Spielfeld. Es wird davon ausgegangen, dass die Schiffspositionen rein zufällig ausgewählt wurden, d.h. dass zu Spielbeginn eine Flotte aus der Menge aller möglichen Flotten ausgewählt wird. Jede mögliche Flotte hat hierbei die gleiche Wahrscheinlichkeit, ausgewählt zu werden.
Natürlich sind die Schiffspositionen dem Spieler nicht bekannt.
Die Gesamtheit aller gewählten Schiffspositionen wird auch die Flotte genannt.
Nun kann der Spieler anfangen, auf bestimmte Positionen auf dem Spielfeld, auch Zellen genannt, zu schießen.
Da Schiffe überlappen können, können mit jedem Schuss mehrere Schiffe, also Teilmengen der Flotte getroffen werden.
Nach jedem Schuss erfährt der Spieler von seinem Gegner, welche Teilmenge der Flotte er getroffenen hat.
Falls keine Schiffe getroffen wurden ist diese Menge natürlich leer.
Ein Schiff gilt als versenkt, sobald es getroffen wurde.
Sobald alle Schiffe versenkt wurden, ist das Spiel beendet.
Das Ziel des Spielers ist es, mit möglichst wenig Schüssen alle Schiffe zu versenken.

\subsection{Inhalt der Ausarbeitung}

Der erste Teil dieser Ausarbeitung beschäftigt sich mit der Bewertung einer gegebenen Strategie, d.h. mit der Berechnung der erwarteten Anzahl an Schüssen, die mit einer Strategie zum versenken der platzierten Flotte gebraucht werden.
Da diese Berechnung sehr aufwändig ist, wird ein Monte-Carlo Algorithmus zur näherungsweisen Berechnung beschrieben, implementiert und ausgewertet.

Im zweiten Teil werden unter den zuvor etablierten Rahmenbedingungen verschiedene Strategien vorgestellt. Dabei können sich diese von denen in der Bachelorarbeit vorgestellten unterschieden, da es sich um andere Rahmenbedingungen handelt. Zu den Strategien zählen dünne und volle Gitter, klassisches und Quasi Monte Carlo und die optimale Maximum Strategie.

Dritter Teil (Implementierung + Vergleich) TODO 

\section{Formalisierung des Spielprinzips}

\begin{definition}
Sei
\[
C_{all} \coloneqq \{1, \dots, N\}^d
\]
die Menge aller Zellen des Spielfeldes mit jeweils $N$ Zellen in $d$ Dimensionen.
\end{definition}

\begin{satz}
Dann gilt:
\[
|C_{all}|=N^d
\]
\end{satz}

\begin{definition}
Seien $c, c' \in C_{all}$ Zellen.
Dann ist
\[
c \leq c' \Leftrightarrow \forall i \in \{1, \dots, d\} \colon c_{i} \leq c'_{i} 
\]
\end{definition}

\begin{definition}
Sei $c_{min} \in C_{all}$ und $c_{max} \in C_{all}$ mit $c_{min} \leq c_{max}$.
\[
l=(c_{min}, c_{max})
\]
eine mögliche Schiffsposition (location), welche mithilfe einer minimalen Ecke $c_{min}$ und einer maximalen Ecke $c_{max}$ bestimmt wird.
\end{definition}

\begin{definition}
Sei 
\[
L_{all} \coloneqq
\{
(i, j) \in C_{all} \times C_{all}
\mid
i \leq j
\}
\] die Menge aller möglichen Schiffspositionen.
\end{definition}

\begin{satz}
Dann gilt:
\[
|L_{all}|=\left(\frac{(N+1) N}{2}\right)^d
\]
\end{satz}

\begin{proof}
Da das Spielfeld in jeder Dimension die Größe $N$ hat, gilt in jeder Dimension $d$ für gültige Start- und Endkoordinaten eines Schiffes $c_d$ und $c'_d$ mit $c_d \leq c'_d$ folgendes:
\begin{align}
\begin{split}
&|\{(c_d, c'_d) \in N \times N \mid c_d \leq c'_d\}|\\
&=|\{(i, j) \in N \times N \mid i \leq j\}|\\
&=\sum_{i=1}^N N - i + 1\\
&=N + \sum_{i=1}^{N-1} i\\
&=\sum_{i=1}^{N} i\\
&= \frac{(N + 1) N}{2}
\nonumber
\end{split}
\end{align}
Daher gibt es genau
\[
\left(\frac{(N+1) N}{2}\right)^d
\]
gültige Schiffspositionen, da für jede Schiffsposition für jede Dimension gültige Start- und Endkoordinaten gewählt werden müssen, also genau $d$ gültige Start- und Endkoordinaten.
\end{proof}

\begin{definition}
Sei 
\[
K \coloneqq \{1, \dots, |L_{all}|\}
\]
die Menge aller möglichen Flottengrößen.
\end{definition}

\begin{definition}
Sei $k \in K$ die Größe der Flotte.
Dann ist
\[
F_{all,k} \coloneqq\{F \in \mathcal{P}(L_{all}) \mid |F| = k\}
\]
die Menge aller Flotten der Größe $k$.
\end{definition}

\begin{satz}
Sei $k \in K$ die Größe der Flotte.
Dann gilt:
\[
|F_{all,k}|=\binom{|L_{all}|}{k}
\]
\end{satz}

\begin{definition}
Dann ist
\[
F_{all} \coloneqq \bigcup_{k=1}^{|L_{all}|} F_{all,k} = \mathcal{P}(L_{all}) \setminus \{\emptyset\}
\]
\end{definition}

\begin{satz}
Dann gilt:
\[
|F_{all}|=2^{|L_{all}|} - 1
\]
\end{satz}

\begin{proof}
\begin{align}
\begin{split}
&|F_{all}|=\sum_{k=1}^{|L_{all}|} |F_{all,k}|\\
=&\sum_{k=1}^{|L_{all}|} \binom{|L_{all}|}{k} \\
=&\left( \sum_{k=0}^{|L_{all}|} \binom{|L_{all}|}{k} \right) - 1 \\
=&2^{|L_{all}|} - 1
\end{split}
\end{align}
\qed
\end{proof}


\begin{definition}
Sei $F \in F_{all}$ und $c \in C_{all}$.
Dann ist 
\begin{align}
&\hit:F_{all} \times C_{all} \rightarrow \mathcal{P}(L_{all}) \quad mit \nonumber\\
&\hit(F, c)\mapsto \{(c_{min}, c_{max}) \in F \mid c_{min} \leq c \leq c_{max}\} \nonumber
\end{align}
die Treffer-Funktion, welche angibt, welche Schiffe bei einem Schuss auf Zelle $c$ getroffen wurden, falls $F$ die vom Gegner platzierte Flotte ist. In anderen Worten, die Menge der Schiffe aus Flotte $F$, die die Zelle $c$ belegen.
\end{definition}

\subsection{Schuss-Strategien}

\begin{definition}
Eine Funktion der Form
\begin{align}
&\strat:\mathcal{P}(C_{all}) \rightarrow C_{all} \nonumber
\end{align}
wird Strategiefunktion genannt. Diese wählt die nächste Zelle aus, auf die geschossen werden soll, abhängig davon, auf welche Zellen bereits geschossen wurden.
\end{definition}

\begin{definition}
Sei
\[
\strategies  \coloneqq \{ \strat:P(C_{all}) \rightarrow C_{all} \}
\]
die Menge an allen Strategiefunktionen.
\end{definition}

\begin{definition}
Sei $F\in F_{all}$ die gewählte Flotte und $S \in P(C_{all})$ die Menge an bereits beschossenen Zellen.
Dann ist:
\[
\sunk(F, S) \Leftrightarrow \bigcup_{c \in S} \hit(F, c) = F
\]
wahr, falls alle Schiffe aus der Flotte $F$ versenkt wurden.
\end{definition}


\begin{definition}
Sei $F\in F_{all}$ die gewählte Flotte und $S$ die Menge an bereits beschossenen Zellen.
Sei außerdem $\strat \in \strategies$ die verwendete Schuss-Strategie.
Dann ist
\begin{align}
&\shot_count(F, S, \strat)=
& \begin{cases} 
  	0& ,\sunk(F, S) \\
      \shot_count(F, S \cup \strat(S), \strat) + 1 & ,sonst
   \end{cases}
\nonumber
\end{align}
die Anzahl an Schüssen, die benötigt werden, um alle Schiffe der Flotte $F$ mit der Schuss-Strategiefunktion $\strat$ zu versenken, falls bereits auf die Zellen der Menge $S$ geschossen wurde.

Außerdem ist dann
\begin{align}
&\shot_count(F, \strat)=\shot_count(F, \emptyset, \strat)
\nonumber
\end{align}
die Anzahl an Schüssen, die benötigt werden, um alle Schiffe der Flotte $F$ mit der Schuss-Strategiefunktion $\strat$ zu versenken.
\end{definition}

\subsection{Zufallsvariablen}

\begin{definition}
Sei $\Omega \subseteq F_{all}$ eine Menge an Flotten.
\end{definition}

\begin{definition}
Sei $A \subseteq \Omega$ das Ereignis, dass bei einem Auswählen einer Flotte aus $\Omega$, die Flotte ebenfalls in der Menge $A$ liegt.
Dann ist
\[
P(A) = \frac{|A|}{|\Omega|}
\]
die Wahrscheinlichkeit, dass eine Flotte aus der Menge $A$ ausgewählt wird.
\end{definition}

Dann ist $(\Omega, P)$ ein endlicher Laplacescher W-Raum und P eine Gleichverteilung.

\begin{definition}
Sei $\strat \in \strategies$ die verwendete Schuss-Strategie.
Sei $\omega$ die ausgewählte Flotte aus $\Omega$.
Dann ist
\begin{align}
\mathbf{T}_{\strat}(\omega) \coloneqq \shot_count(\omega, \strat)
\nonumber
\end{align}
eine Zufallsvariable, die die Anzahl an Schüssen zum versenken einer Flotte $\omega$ bezeichnet.
\end{definition}

\begin{satz}
Sei $\strat \in \strategies$ die verwendete Schuss-Strategie.
Dann ist
\begin{align}
E_\Omega(\mathbf{T}_{\strat})=\frac{1}{|\Omega|} \sum_{\omega \in \Omega} \shot_count(\omega, \strat)
\nonumber
\end{align}
der Erwartungswert von $\mathbf{T}_{\strat}$ in der Menge $\Omega$.
\end{satz}

\begin{proof}
\begin{align}
\begin{split}
E_\Omega(\mathbf{T}_{\strat})&=\sum_{\omega \in \Omega} P(\omega) \shot_count(\omega, \strat)\\
&=\frac{1}{|\Omega|} \sum_{\omega \in \Omega} \shot_count(\omega, \strat)
\nonumber
\end{split}
\end{align}
\qed
\end{proof}

\begin{satz}
Sei $\strat \in \strategies$ die verwendete Schuss-Strategie.
Dann ist
\begin{align}
E_{F_{all}}(\mathbf{T}_{\strat})=\frac{1}{|F_{all}|} \sum_{k=1}^{|L_{all}|}\sum_{F\in F_{all,k}} \shot_count(F, \strat)
\nonumber
\end{align}
der Erwartungswert von $\mathbf{T}_{\strat}$ in der Menge $F_{all}$.
\end{satz}

\begin{proof}
\begin{align}
\begin{split}
&E_{F_{all}}(\mathbf{T}_{\strat})\\
=&\frac{1}{|F_{all}|} \sum_{F\in F_{all}} \shot_count(F, \strat)\\
=&\frac{1}{|F_{all}|} \sum_{k=1}^{|L_{all}|}\sum_{F\in F_{all,k}} \shot_count(F, \strat)
\nonumber
\end{split}
\end{align}
\qed
\end{proof}

\begin{definition}
Sei $k \in K$ die Flottengröße.
Dann ist
\[
g_k=\frac{|F_{all,k}|}{|F_{all}|}=\frac{\binom{|L_{all}|}{k}}{2^{|L_{all}|} - 1}
\]
der Anteil an Flotten der Größe $k$ an der Menge aller Flotten. Dieser Faktor wird später als Gewichtungsfaktor für den Erwartungswert der Flottengröße $k$ verwendet.
\end{definition}

\begin{satz}
Sei $\strat \in \strategies$ die verwendete Schuss-Strategie.
Dann ist
\begin{align}
E_{F_{all}}(\mathbf{T}_{\strat})=\sum_{k=1}^{|L_{all}|} g_k E_{F_{all,k}}(\mathbf{T}_{\strat})
\nonumber
\end{align}
der Erwartungswert von $\mathbf{T}_{\strat}$ in der Menge $F_{all}$.
\end{satz}

\begin{proof}
\begin{align}
\begin{split}
&E_{F_{all}}(\mathbf{T}_{\strat})\\
=&\frac{1}{|F_{all}|} \sum_{k=1}^{|L_{all}|}\sum_{F\in F_{all,k}} \shot_count(F, \strat)\\
=&\sum_{k=1}^{|L_{all}|} \frac{1}{|F_{all}|} \sum_{F\in F_{all,k}} \shot_count(F, \strat)\\
=&\sum_{k=1}^{|L_{all}|} \frac{1}{|F_{all}|} |F_{all,k}| \sum_{F\in F_{all,k}} \frac{1}{|F_{all,k}|} \shot_count(F, \strat)\\
=&\sum_{k=1}^{|L_{all}|} \frac{|F_{all,k}|}{|F_{all}|} E_{F_{all,k}}(\mathbf{T}_{\strat})\\
=&\sum_{k=1}^{|L_{all}|} g_k E_{F_{all,k}}(\mathbf{T}_{\strat})
\nonumber
\end{split}
\end{align}
\qed
\end{proof}

\section{Berechnung des Erwartungswertes mithilfe eines MC-Algorithmus}

Um zu bewerten, wie gut eine Strategiefunktion $\strat$ abschneidet, kann ein Monte-Carlo Algorithmus implementiert werden, der für jede Flottengröße $k \in K$ mit einer Stichprobe den Wert von $E_{F_{all,k}}(\mathbf{T}_{\strat})$ abschätzt und am Ende alle diese Werte zum gesamtem Erwartungswert $E_{F_{all}}(\mathbf{T}_{\strat})$ kombiniert.

Um den Wert von $\shot_count(F, \strat)$ für eine gegebene Flotte $F$ zu berechnen, kann folgende, (naive) Funktion verwendet werden:

\begin{tcolorbox}
	\begin{algorithmic}[H]
		\Function{shot\_count}{$F, \strat$}{:}
		\State $S=\emptyset$;
		\State $\shot_count=0$;
		\While{$\neg \sunk(F, S)$}
		\State $next\_target=\strat(S)$;
		\State $S=S \cup \hit(F, next\_target)$;
		\State $\shot_count=\shot_count+1$;
		\EndWhile
		\State\Return $\shot_count$;
		\EndFunction
	\end{algorithmic}
\end{tcolorbox}

Mithilfe dieser Funktion werden dann Werte für viele, zufällig gewählte Flotten zu einem Erwartungswert der Stichprobe kombiniert:

\begin{tcolorbox}
	\begin{algorithmic}[H]
		\Function{evaluate\_strategy}{$n, \strat$}{:}
		\State //Generiere $n$ verschiedene Flotten;
		\State $fleets = generate\_random\_fleets(n)$;
		\For{$i=1,\dots,n$}
		\State $shots= shot\_count(fleets[i], \strat)$;
		\State $P(\mathbf{X}^F_{\strat} = shots)+=\frac{1}{n}$;
		\EndFor
		\EndFunction
	\end{algorithmic}
\end{tcolorbox}

\subsection{Eine bessere Alternative}


\begin{definition}
Eine Funktion der Form
\begin{align}
&I_{\strat}:C_{all} \rightarrow \{1, \dots, |C_{all}|\} \nonumber
\end{align}
wird Indexfunktion einer Strategie genannt. Diese weißt jeder Zelle des Spielfeldes den Wert zu, zu welchem die Zelle getroffen wird. Der Vorteil ist, dass diese Werte einmalig für eine Strategie berechnet werden müssen und dann abgespeichert werden können.
\end{definition}

\begin{definition}
Sei $\strat \in \strategies$ die verwendete Schuss-Strategie.
Sei $\omega$ das ausgewählte Schiff aus $L_{all}$.
Dann ist
\begin{align}
\mathbf{X}^L_{\strat}(\omega) = \min_{c \in cells(\omega)} I_{\strat}(c)
\nonumber
\end{align}
eine Zufallsvariable, die die Anzahl an Schüssen zum versenken einer beliebig gewählten Schiffes $\omega$ aus $L_{all}$ bezeichnet.
\end{definition}

\begin{definition}
Sei $\strat \in \strategies$ die verwendete Schuss-Strategie.
Sei $\omega$ die ausgewählte Flotte aus $F_{all}$.
Dann ist
\begin{align}
\mathbf{X}^F_{\strat}(\omega) = \max_{l \in \omega} \mathbf{X}^L_{\strat}(l)
\nonumber
\end{align}
eine Zufallsvariable, die die Anzahl an Schüssen zum versenken einer beliebig gewählten Flotte $\omega$ aus $F_{all}$ bezeichnet.
\end{definition}

\begin{tcolorbox}
	\begin{algorithmic}[H]
		\Function{calculate\_turns}{$n, \strat$}{:}
		\State //Generiere $n$ verschiedene Schiffe;
		\State $ships = generate\_random\_ships(n)$;
		\State $turns = [1, ..., |C_{all}|]$;
		\State $sum=0$;
		\For{$i=1,\dots,n$}
		\State $min= \min_{c \in cells(ships[i])} I_{\strat}(c)$;
		\State $turns[min]++$;
		\EndFor
		\State\Return $turns$;
		\EndFunction
	\end{algorithmic}
\end{tcolorbox}

\begin{tcolorbox}
	\begin{algorithmic}[H]
		\Function{evaluate\_strategy}{$n, \strat$}{:}
		\State $turns = calculate\_turns(n, \strat)$;
		\State $ship\_sum=0$;
		\For{$i=1,\dots,|C_{all}|$}
		\State $ships=turns[i]$;
		\State $ship\_sum += ships$;;
		\State $P(\mathbf{X}^L_{\strat} = i)=\frac{ships}{n}$;
		\State $P(\mathbf{X}^L_{\strat} \leq i)=\frac{ship\_sum}{n}$;
		\State $P(\mathbf{X}^F_{\strat} \leq i)=\frac{2^{ship\_sum}}{2^{n} - 1}$;
		\State $P(\mathbf{X}^F_{\strat} = i)=(P(\mathbf{X}^F_{\strat} = i) - P(\mathbf{X}^F_{\strat} \leq i - 1))$;
		\EndFor
		\EndFunction
	\end{algorithmic}
\end{tcolorbox}

Hier werden aus den Ergebnissen von einer Stichprobe mit $n$ Schiffen das Ergebnis für alle möglichen Flotten, die aus den Schiffen der Stichprobe erstellt werden können, berechnet. Die Anzahl dieser Flotten beträgt $2^n - 1$.

\section{Strategien}

Eine Strategie gibt eine Folge von Schüssen vor, die auf das Spielfeld abgegeben werden. Dabei unterscheiden sich die Strategien in der Auswahl ihrer Schüsse. Eine mögliche Vorgehensweise wäre die zufällige Wahl der nächsten zu beschießenden Zelle. Außerdem ist es interessant zu sehen wie diskrepanzminimierende Diskretisierungsverfahren als mögliche Schussstrategie abschneiden. Dies wurde erstmals in \cite{M13} näher betrachtet. Dabei werden die von diesen Verfahren genutzten Stützstellen als Schussfolgen interpretiert. Um Strategien bewerten zu können benötigt man den Vergleich zu einer optimalen Schussstrategie. Diese wurde ebenfalls in \cite{M13} vorgestellt allerdings für ineffizient erklärt. Es bleibt zu klären, ob die optimale Strategie mit der hier vorgestellten Monte-Carlo Herangehensweise effizienter umzusetzen ist und zu Vergleichen auch in höheren Dimensionen herangezogen werden kann.

Sei für alle folgenden Strategien $F\in F_{all}$ die aktuell vorliegende unbekannte Flotte.

\subsection{Maximum Strategie oder optimale Strategie}

Diese in \cite{M13} Abbildung 3.15 vorgestellte Strategie ist optimal, da sie immer an die Position schießt an der sie zu jedem Zeitpunkt am meisten Schiffe auf einmal treffen und damit versenken kann. Dazu wird zunächst eine Anfangsverteilung berechnet. Jeder Zelle wird die Anzahl der darauf liegenden Schiffe zugewiesen. Der nächste Schuss wird auf die Zelle mit der größten Zahl abgegeben. Schließlich wird die Verteilung aktualisiert um die gerade versenkten Schiffe.

Die optimale Strategie kann durch folgenden Algorithmus in Pseudo-Code beschrieben werden:

\begin{tcolorbox}
	\begin{algorithmic}
		\Function{max\_strat}{$S$}{$:cell$}
		\State $cell = null$;
		\State $max = 0$;
		\State $temp = 0$;
		\For{\textbf{each } $c\in C_{all}$}
		\State $temp=\left|\hit(F,c)\right|$;
		\If{$temp>max$}
		\State $max=temp$;
		\State $cell = c$;
		\EndIf
		\EndFor
		\State \Return $cell$;
		\EndFunction
	\end{algorithmic}
\end{tcolorbox}

In \cite{M13} wurde bereits angesprochen, dass dieser Algorithmus sehr hohe Kosten hat. Es wird über alle Zellen des Spielfelds iteriert, das heißt wir haben $N^d$ Schleifendurchläufe pro Aufruf der Strategie. In jedem dieser Durchläufe wird die $\hit$ Funktion aufgerufen. Diese muss für jedes Schiff aus der aktuellen Flotte $F$ prüfen ob dieses die momentan anvisierte Zelle $c$ enthält. Diese Überprüfung kann mittels zwei arithmetischen Vergleichen (Erinnerung: ein Schiff wird als Tupel $(c_{min},c_{max})$ gespeichert) erledigt werden. Damit haben wir für den Aufruf der $\hit$ Funktion Kosten von $O(2\cdot |F|)$, bzw. $O(F)$. Somit betragen sich die gesamten Kosten der optimalen Strategie auf $O(N^d\cdot F)$. 

\subsection{Volle Gitter Strategie}

\subsubsection{Allgemeines zu Gittern}

In Abschnitt 2.2.1 von \cite{M13} werden die hierarchischen Stützstelleninkremente $H_{\underline{m}}$ für Gitter vorgestellt, welche sich von den in \cite{M13} Abbildung 2.2 dargestellten Hütchenfunktionen ableiten. Diese Inkremente bzw. ihre Kombinationen werden sowohl für dünne als auch volle Gitter genutzt. Dabei halbiert sich die Maschenweite mit steigendem Level in einer beliebigen Dimension. Dies ist schön zu sehen in Abbildung \ref{fig:gitter01} (ähnlich aufgebaut wie \cite{P10} Figure 2.5).

Es lassen sich durch Kombinationen von Inkrementen Gitter erzeugen. Unsere Schussfolge ergibt sich dann aus den abgearbeiteten Inkrementen bzw. den darin enthaltenen Stützstellen. 

\begin{figure}
	\centering
	\resizebox{90mm}{!}{\input{figures/Grids_01.pdf_tex}}
	\caption{Verdeutlichung Halbierung der Maschenweite}
	\label{fig:gitter01}
\end{figure}


Noch zu beachten ist, dass Level bei Gittern die Maschenweite und damit die Genauigkeit angeben. Normalerweise legt man ein Level fest und erzeugt sich dann aus Inkrementen ein Gitter eben dieses Levels. In unseren Fall arbeiten wir uns durch die Inkremente bei steigendem Level aufsteigend durch um immer genauere Gitter zu erzeugen, bis wir alle Schiffe der Flotte getroffen haben.

\subsubsection{Mathematische Grundlagen der vollen Gitter}

In \cite{M13} Definition 2.14 wird beschrieben, wie sich volle Gitter aus den Inkrementen zusammensetzen. Dabei ist lediglich zu beachten, dass die Indeces eines Inkrements $H_{\underline{m}}$ nur kleiner dem Level $n$ sein müssen.  Zur Wiederholung:

\begin{definition}
	Seien $d,n\in\mathbb{N}$ und $d,n>1$. Ein $d$-dimensionales volles Gitter $G$ mit Level $n$ wird durch Kombination der Inkremente $H_{\underline{m}}$ mit $\underline{m}<(n,\dots,n)$ erzeugt:
	\begin{equation}
		G=\biguplus_{m_1=1}^n\biguplus_{m_2=1}^n\dots \biguplus_{m_d=1}^n H_{\underline{m}}
	\end{equation}
\end{definition}

Diese Definition kann anhand Abbildung \ref{fig:gitter02} verdeutlicht werden. Darin sind einige mögliche Kombinationen von Inkrementen markiert, die ein volles Gitter von Level 2 bis 4 bilden.

\begin{figure}
	\subfigure[Volles Gitter Level 2]{\resizebox{50mm}{!}{\input{figures/Grids_02.pdf_tex}}}
	\subfigure[Volles Gitter Level 3]{\resizebox{50mm}{!}{\input{figures/Grids_03.pdf_tex}}}
	\subfigure[Volles Gitter Level 4]{\resizebox{50mm}{!}{\input{figures/Grids_04.pdf_tex}}}
	\caption{Inkremente von zweidimensionalen vollen Gittern}
	\label{fig:gitter02}
\end{figure}

Wie man gut an Abbildung \ref{fig:gitter02} sehen kann können volle Gitter kleineren Levels für größere genutzt werden. Für eine Schussfolge heißt das man startet bei dem Gitter das nur aus einem Punkt besteht und arbeitet sich zu Gittern höheren Levels vor. Damit bei der Schussfolge keine Wiederholungen auftreten sollten nur neu hinzugekommene Punkte bei Erhöhen des Levels beachtet werden. Im zweidimensionalen entspricht dies gerade Quadraten die bei Erhöhen des Levels um eine Längeneinheit pro Kante wachsen. Dabei wird dann lediglich die hinzukommende Schicht an Punkten betrachtet. Damit lässt sich einer Schussfolge bereits eine grobe Struktur geben. 

Außerdem ist an Abbildung \ref{fig:gitter02} zu sehen, dass in den neu hinzukommenden Punkten mindestens eine neue Koordinaten verwendet wird. Das heißt wir können pro neuer Schicht zu Beginn einmalig die neuen Koordinaten berechnen und anschließend diese einfach aufsteigend miteinander zu Punkten kombinieren. Dabei können Schüsse wie Zahlen in einem Zahlensystem betrachtet werden. Wir denken uns die verwendeten Koordinaten aus denen Schüsse kombiniert werden können als Ziffern eines Zahlensystems. Dabei sind Koordinaten eines höheren Levels größer einzuordnen als die eines niedrigeren Levels. Außerdem werden die Koordinaten innerhalb eines Levels von klein nach groß sortiert. Damit ergibt sich eine absolute Ordnung der Punkte und Schüsse dieser Strategie lassen sich durch Hochzählen der Zahlen des Zahlensystems bestimmen.

Von diesen Koordinaten gibt es maximal $N$ viele. Diese werden im nachfolgenden Algorithmus in $K$ gespeichert.

\subsubsection{Algorithmus für volle Gitter Strategie}

Der Algorithmus soll eine feste Menge an möglichen Koordinaten $K$ für ein Level $L$ haben. Davon werden alle Linearkombinationen durch inkrementieren der Koordinaten (vgl. absolute Ordnung) gebildet. Sobald alle Kombinationen erreicht wurden wird $K$ um die neuen Koordinaten des nächsten Levels erweitert. Diese grobe Struktur lässt sich folgendermaßen darstellen:

\begin{tcolorbox}
	\begin{algorithmic}
		\Function{full\_grid\_strat}{$S,K,L$}{$:cell$}
		\If{$K=\emptyset$}
		\State \textsc{initiate\_Coordinates}()
		\EndIf
		\If{$S[last\_index]=(max\_coord,\dots,max\_coord)$}
		\State \textsc{add\_New\_Coordinates}()
		\EndIf
		\If{$S=\emptyset$}
		\State $lastIndeces = (0,\dots,0)$
		\State \Return $(K[0],\dots,K[0])$
		\Else
		\State \textsc{increment\_Coordinates}()
		\EndIf
		\EndFunction
	\end{algorithmic}
\end{tcolorbox}

Im folgenden werden die Abschnitte und Funktionen des Algorithmus genauer betrachtet und ihre Komplexität untersucht.

\begin{tcolorbox}
	\begin{algorithmic}
		\Function{initiate\_Coordinates}{}{}
		\State $K=\left[\left\lfloor\frac{N}{2}\right\rfloor\right]$
		\State $max\_coord=K\left[0\right]$
		\State $L=0$
		\EndFunction
	\end{algorithmic}
\end{tcolorbox}

Mit dieser Funktion wird die erste Koordinate initialisiert. Wenn $K$ leer haben wir noch keine Koordinate gespeichert und daher auch noch keinen Schuss abgegeben. Die erste Koordinate, die für das erste volle Gitter benutzt wird ist gerade die Größe des Spielfelds $N$ halbiert. Falls $N$ ungerade ist abgerundet. $max\_coord$ enthält die maximal mögliche Koordinate. $L$ beschreibt das maximale Level von dem unsere Koordinaten in $K$ stammen. Gesamte Komplexität $O(1)$.

\smallskip
\hrule
\smallskip
\begin{tcolorbox}
	\begin{algorithmic}
		\Function{add\_New\_Coordinates}{}
		\State $min=\left\lceil\frac{K[2^L-1]}{2}\right\rceil$
		\State $K_{old}=K\left[2^L-1\dots 2^{L+1}-2\right]$
		\For{\textbf{each } $k\in K_{old}$}
		\If{$K[last\_index]!=k-min$}
		\State $K.add(k-min)$
		\EndIf
		\State $K.add(k+min)$
		\EndFor
		\State $L++$
		\State $max\_coord=K\left[2^{L+1}-2\right]$
		\EndFunction
	\end{algorithmic}
\end{tcolorbox}

Wenn der letzte Schuss nur aus den größtmöglichen Koordinaten besteht haben wir bereits zuvor alle anderen Kombinationsmöglichkeiten ausgenutzt und müssen nun die Menge an Koordinaten $K$ erweitern. Dazu werden gerade die Werte als Koordinaten hinzugefügt, die in den Zwischenräumen zweier vorheriger Koordinaten liegen (Vgl. Abbildung \ref{fig:gitter01}). 

Da wir in unserer Herangehensweise diskret arbeiten kann es sein, dass das Halbieren nicht immer aufgeht daher muss hier gerundet werden. Es wird aufgerundet, da sonst der Fall eintreten kann, das Punkte nicht erreicht werden können. Dabei wird durch eine Abfrage gewährleistet das keine gleichen Koordinaten aufgenommen werden. Zum Schluss wird die Menge der Koordinaten aktualisiert und das Level erhöht. $K$ kann maximal um $\left\lceil\frac{N}{2}\right\rceil$ viele neue Koordinaten erweitert werden. Das ist gleichzeitig auch die maximal mögliche Größe von $K_{old}$. Also haben wir für die Schleife $O(N)$. 


Der Zugriff auf den letzten Index von $S$ kann mit einer double-linked list oder einem Array auf $O(1)$ reduziert werden und der Vergleich im If benötigt $d$ Schritte. Daher ist die Komplexität $O\left(N+d\right)$.

\smallskip
\hrule
\smallskip

\begin{tcolorbox}
	\begin{algorithmic}
		\If{$S=\emptyset$}
		\State $lastIndeces = (0,\dots,0)$
		\State \Return $(K[0],\dots,K[0])$
		\Else
		\State $\dots$
		\EndIf
	\end{algorithmic}
\end{tcolorbox}

Hier wird geprüft ob wir uns am Anfang des ersten Gitters befinden und der Schuss wird entsprechend ausgegeben. Außerdem wir ein Array, das die Indeces der verwendeten Koordinaten Speichert initialisiert. Dies erleichtert uns später das Finden des nächsten Schusses. Wir bewegen uns in $O(1)$.

\smallskip
\hrule
\smallskip

\begin{tcolorbox}
	\begin{algorithmic}
		\Function{increment\_Coordinates}{}
		\State $s=S[last\_index]$
		\State $count =d$
		\While{True}
		\If{$s[count]=max\_coord$}
		\State $s[count]=K[0]$
		\State $lastIndeces[count]=0$
		\State $count--$
		\Else
		\State $s[count] = K[lastIndeces[count]+1]$
		\State $lastIndeces[count]++$
		\State\textbf{break}
		\State\Return $s$
		\EndIf
		\EndWhile
		\EndFunction
	\end{algorithmic}
\end{tcolorbox}

In diesem Abschnitt wird das Inkrementieren im gedachten Zahlensystem realisiert. Dabei wird von der letzten Ziffer nach vorne gehend geprüft welche davon den aktuell maximal möglichen Wert hat. Diese Ziffern müssen beim zählen dann zurück auf Null bzw. die kleinste Koordinate $K[0]$ springen. Sobald die erste Ziffer erreicht wird, die nicht den maximal Wert hat kann diese um eins erhöht werden. Die restlichen Ziffern, falls vorhanden, bleiben gleich. 

Die Schleife kann maximal $d$ Schritte durchführen. Es wurde bereits vorher ausgeschlossen, dass $s$ nur aus maximalen Koordinaten bestehen kann, also springen wir auf jeden Fall irgendwann innerhalb der $d$ Schritte in den else Teil. 

Der absolute worst case wäre wenn $d-1$ mal der if Teil und einmal der else Teil aufgerufen werden. Das heißt dieser Abschnitt liegt in $O(d)$.

\subsubsection{Komplexität volle Gitter Strategie}

Wir können somit die Zeitkomplexitäten der Abschnitte zusammenfassen und erhalten eine gesamte Komplexität für den Algorithmus von $O(1)+O(N+d)+O(1)+O(d)=O(N+d)$.


Bezüglich der Speicherplatz Komplexität haben wir für $K$ $O(N^2)$, da für die Anzahl der Elemente in $K$ gilt:

\begin{equation}
\sum_{j=1}^{\left\lceil\frac{N}{2}\right\rceil}j=\frac{\left\lceil\frac{N}{2}\right\rceil\cdot\left(\left\lceil\frac{N}{2}\right\rceil+1\right)}{2}\in O\left(N^2\right)
\end{equation}


Für $S$ gilt allgemein $O(N^d)$ wenn man jeden Schuss speichert, allerdings benötigt diese Strategie nur den letzten Schuss. Wenn man sich also entscheidet nur diesen zu speichern hat man $O(1)$. Damit also gesamt Speicherplatz Komplexität von $O(N^2)$.


\subsection{Dünne Gitter Strategie}

\subsubsection{Mathematische Grundlagen der dünnen Gitter}

In \cite{M13} Definition 2.15 wird beschrieben, wie sich dünne Gitter aus den Inkrementen zusammensetzen. Zur Wiederholung:

\begin{definition}
	Seien $d,n\in\mathbb{N}$ und $d,n>1$. Ein $d$-dimensionales dünnes Gitter $G$ mit Level $n$ wird durch Kombination der Inkremente $H_{\underline{m}}$ mit $|\underline{m}|_1\leq n+d-1$ erzeugt:
	\begin{equation}
	G=\biguplus_{|\underline{m}|_1\leq n+d-1}^n H_{\underline{m}}
	\end{equation}
\end{definition}

Wobei für die verwendete Norm gilt (Vgl. \cite{P10} Abschnitt 2.1 Basics):

\begin{definition}
Die  $|\cdot|_1$-Norm ist definiert als:
	\begin{equation}
	|\underline{m}|_1:=\sum_{j=1}^d m_j
	\end{equation}
\end{definition}


Die Definition des dünnen Gitters kann anhand Abbildung \ref{fig:gitter03} verdeutlicht werden. Darin sind einige mögliche Kombinationen von Inkrementen markiert, die ein dünnes Gitter von Level 1 bis 3 bilden.

\begin{figure}
	\subfigure[Dünnes Gitter Level 1]{\resizebox{50mm}{!}{\input{figures/Grids_02.pdf_tex}}}
	\subfigure[Dünnes Gitter Level 2]{\resizebox{50mm}{!}{\input{figures/Grids_05.pdf_tex}}}
	\subfigure[Dünnes Gitter Level 3]{\resizebox{50mm}{!}{\input{figures/Grids_06.pdf_tex}}}
	\caption{Beispiele für zweidimensionale dünne Gitter}
	\label{fig:gitter03}
\end{figure}

\subsubsection{Algorithmus für dünne Gitter}

In Abbildung \ref{fig:gitter03} ist zu sehen, dass sich im zweidimensionalen Fall bei steigendem Level der Gitter immer eine Diagonale dazukommt. Für den Algorithmus der dünne Gitter Strategie benötigen wir zusätzlich noch ein $S'$ und anstelle von $K$ ein $K'$. Dabei funktioniert $S'$ ganz analog zu $S$ wobei Tupel aus zwei $d$-dimensionalen Vektoren gespeichert werden. Der erste um die Zelle zu beschreiben und der zweite um das Level aus der die korrespondierende Koordinate stammt zu definieren. $K'$ speichert nun die Koordinaten nach Leveln getrennt in verschiedenen Listen oder Arrays und nicht wie $K$ noch alles in einer Liste/Array.

Die Funktionsweise des folgenden Algorithmus leitet sich von mit dünne Gittern einhergehenden Bedingung an die Inkremente ab. Dabei ist zu beachten, dass $|\underline{m}|_1\leq n+d-1$ gilt. Bei konstanter Dimension erhöht sich pro dazukommende Diagonale um das nächste Gitter zu erzeugen jeweils nur $n$ um eins. Das heißt von den Indeces der Inkremente kann sich maximal eines auf ein neues Level erhöhen damit die Bedingung weiterhin erhalten bleibt. Also erarbeiten wir ausgehend von den alten Schüssen  die neuen, indem einzelne Koordinaten um ein Level erhöht werden. Dabei ist zu beachten, dass innerhalb eines neuen Levels es mehrere mögliche Koordinaten gibt. 


\begin{tcolorbox}
	\begin{algorithmic}
		\Function{sparse\_grid\_strat}{$S',K',L$}{$:cell$}
		\If{$K'=\emptyset$}
		\State \textsc{initiate\_Coordinates}()
		\EndIf
		\If{$S'[last\_index][0]=(K'[0][0],\dots,K'[0][0],max\_coord)$}
		\State \textsc{add\_New\_Coordinates}()
		\EndIf
		\If{$S=\emptyset$}
		\State $S'.add((('K[0][0],\dots,K'[0][0]),(0,\dots,0)))$
		\State \Return $(K'[0][0],\dots,K'[0][0])$
		\Else
		\State \textsc{find\_Next\_Shot}()
		\EndIf
		\EndFunction
	\end{algorithmic}
\end{tcolorbox}

Wir werden jetzt nun wieder den Algorithmus in kleinen Teilen näher betrachten und dabei die Funktionsweise und Komplexität genauer beschreiben.

\smallskip
\hrule
\smallskip

\begin{tcolorbox}
	\begin{algorithmic}
		\Function{initiate\_Coordinates}{}
		\State $K'[0]=\left[\left\lfloor\frac{N}{2}\right\rfloor\right]$
		\State $max\_coord=K'\left[0\right][0]$
		\State $L=0$
		\State $countCoord = 0$
		\State $position = 0$
		\EndFunction
	\end{algorithmic}
\end{tcolorbox}

Sollte $K$ leer sein werden die erste Koordinate hinzugefügt und ein paar notwendige Variablen initialisiert. Auf die Funktion der Variablen $countCoord$ und $position$ kommen wir noch später genauer zu sprechen. Die Komplexität ist $O(1)$.

\smallskip
\hrule
\smallskip

\begin{tcolorbox}
	\begin{algorithmic}
		\Function{add\_New\_Coordinates}{}
		\State $min=\left\lceil\frac{K'[L][0]}{2}\right\rceil$
		\State $K'.add([])$
		\For{\textbf{each } $k\in K'[L]$}
		\If{$K'[L+1][last\_index]!=k-min$}
		\State $K'[L+1].add(k-min)$
		\EndIf
		\State $K'[L+1].add(k+min)$
		\EndFor
		\State $L++$
		\State $max\_coord=K'\left[L\right][last\_index]$
		\State $countCoord = 0$
		\State $position = 0$
		\EndFunction
	\end{algorithmic}
\end{tcolorbox}

Dieser Abschnitt beschreibt den Fall, dass neue Koordinaten hinzugefügt werden müssen. Dabei ist die später folgende Iteration so aufgebaut, dass der letzte Schuss eines Koordinatensatzes gerade die Form $(K'[0][0],\dots,K'[0][0],max\_coord)$ hat. Da von vorne nach hinten durch die Dimensionen gezählt wird und $max\_coord$ ebenfalls die letzte Koordinate eines Levels ist. Es wird nun wie zuvor bereits erwähnt eine neue Liste/Array erzeugt, die die Koordinaten des neuen Levels enthalten soll. Die Berechnungen unterscheiden sich dabei nicht von denen bei vollen Gittern. Zum Schluss werden noch einmal $countCoord$ und $position$ gesetzt, da diese wie wir noch sehen werden, sich im Laufe verändert haben und nun zurückgesetzt werden müssen.

Der Vergleich im If ist in $O(d)$. Die Schleife hat $\left\lceil\frac{N}{4}\right\rceil$ Durchläufe, da das gerade die Anzahl der Elemente im vorletzten Level ist. Das letzte Level taucht als Schleife nicht mehr auf, da bis dahin bereits alle Punkte des Spielfelds beschossen sind. Also ist die Schleife in $O(N)$. Davon ausgehend, dass $last\_index$ in $O(1)$ realisiert werden kann hat dieser Abschnitt eine gesamte Komplexität von $O(N+d)$.

\smallskip
\hrule
\smallskip

\begin{tcolorbox}
	\begin{algorithmic}
		\If{$S=\emptyset$}
		\State $S'.add((('K[0][0],\dots,K'[0][0]),(0,\dots,0)))$
		\State \Return $(K'[0][0],\dots,K'[0][0])$
		\Else
		\State $\dots$
		\EndIf
	\end{algorithmic}
\end{tcolorbox}

Der If Abschnitt liegt in $O(1)$ bei Verwendung eines Arrays, ist aber auch nicht weiter relevant, da er nur einmal aufgerufen wird. 

\smallskip
\hrule
\smallskip

\begin{tcolorbox}
	\begin{algorithmic}
		\Function{find\_Next\_Shot}{}
		\State $out = S'[0]$
		\If{$countCoord=0$}
		\State $out[1][position]++$
		\EndIf
		\State $out[0][position]=K[out[1][position]][countCoord]$
		\State $countCoord++$
		\If{$countCoord=K[position].size()$}
		\State $countCoord=0$
		\State $position++$
		\If{$position=d$}
		\State $position=0$
		\State $S'.delete(0)$
		\EndIf
		\EndIf
		\State $S'.add(out)$
		\State\Return $out[0]$
		\EndFunction
	\end{algorithmic}
\end{tcolorbox}

Die Funktion aus dem Else Abschnitt hat zunächst ein paar $O(1)$ Zeilen, da es sich nur um Zugriff oder arithmetische Operationen handelt. Die $size()$ Funktion liegt bei maximalem Level in $O(N)$. Die $delete()$ Funktion ist hier $O(1)$, da das erste Element entfernt wird, was in $O(1)$ geht. Somit hat der Else Teil und damit der gesamte Abschnitt eine Komplexität von $O(N)$.

\subsubsection{Komplexität dünne Gitter Strategie}

Daraus ergibt sich für den gesamte Algorithmus ein Zeitkomplexität von $O(N+d)$. Der Speicherbedarf ist für $K'$ analog zu $K$ von den vollen Gittern, also $O(N^2)$. Was $S'$ angeht werden hier mehrere Schüsse gespeichert und auch verwendet. Bildlich im Zweidimensionalen gesprochen wird zu jedem Zeitpunkt eine komplette Diagonale an Inkrementen und deren dazugehörige Schüsse gespeichert. Nach \cite{GG08} ist die Größenordnung der Anzahl an Punkten für dünne Gitter $O(2^LL^{d-1})$ für Level $L$. Da wir nur eine Diagonale speichern ergibt sich für unsere Menge an Punkten dennoch:

\begin{equation}
2^LL^{d-1}-2^{L-1}(L-1)^{d-1}=2^L\left(L^{d-1}-\frac{(L-1)^{d-1}}{2}\right)\in O(2^LL^{d-1})
\end{equation}

Also haben wir einen Speicherbedarf von $O(N^2+2^LL^{d-1})$. Wir müssen im worst case mindestens auf ein Level von $L=\log_2(N)$, da nur so alle Punkte im Raum überhaupt erst getroffen werden können. Das heißt wir haben eine untere Schranke an den Speicherplatz von $\Omega(N^2+N\log_2(N)^{d-1})$.


\subsection{Monte-Carlo Strategie}

Ähnlich wie Strategien aus der Diskretisierung durch dünne und volle Gitter gewonnen wurden, kann eine Strategie aus der Monte-Carlo Diskretisierung abgeleitet werden. Dabei stellt die Wahl der Stützstellen, die bei Gittern noch nach Gesetzmäßigkeiten ausgewählt wurden, effektiv die Strategie dar, so dass diese Punkte den Schüssen entsprechen. Da bei Monte-Carlo diese Stützstellen zufällig gewählt werden, tun wir dies auch hier für die Monte-Carlo Strategie. 

Diese Strategie kommt einer realen Strategie am nächsten, da die meisten Spieler nach dem Zufallsprinzip vorgehen. Der Algorithmus ist sehr simpel, da lediglich eine zufällige Zelle als Ziel des Schusses gewählt werden muss.

\begin{tcolorbox}
	\begin{algorithmic}
		\Function{monte\_carlo\_strat}{$S$}{$:cell$}
		\State \Return $randomFrom(C_{all}\backslash S)$
		\EndFunction
	\end{algorithmic}
\end{tcolorbox}

Die zufällige Auswahl von Zellen, die nicht bereits beschossen wurden lässt sich leicht realisieren indem man zum Beispiel zuerst alle Zellen durchzählt, wobei auf die erste als erstes geschossen werden soll und die Zellen dann eine Weile miteinander zufällig tauscht. Damit erreicht man eine zufällige Schussfolge.


\subsection{Sobol Strategie}

\subsection{Halton Strategie}

\section{Ergebnisse}

Bestimmt man nun mithilfe der Implementation die Verteilungsfunktionen von $\mathbf{X}^L_{\strat}(\omega)$ und $\mathbf{X}^F_{\strat}(\omega)$, können unterschiedliche Strategien auch über verschiedene Spielfeldgrößen gut miteinander verglichen werden:

% Literaturverzeichnis ------------------------------------------------
\newpage
\bibliographystyle{alphadinLinkLocal}
%\bibliography{literatur} 

\begin{thebibliography}{WB95}
	\providecommand{\url}[1]{\texttt{#1}}
	\expandafter\ifx\csname urlstyle\endcsname\relax
	\providecommand{\doi}[1]{doi: #1}\else
	\providecommand{\doi}{doi: \begingroup \urlstyle{rm}\Url}\fi
	
	\bibitem[WB95]{WB95}
	\textsc{Welch}, G. ; \textsc{Bishop}, G.:
	\newblock An Introduction to the Kalman Filter  / UNC-CH Computer Science.
	\newblock \,Version:\,1995.
	\newblock  \url{http://www.cs.unc.edu/~welch/media/pdf/kalman_intro.pdf}
	(95-041). --
	\newblock Technical Report. --
	\newblock Online--Ressource
	
	\bibitem[M13]{M13}
	\textsc{Mehlbeer}, F.:
	\newblock \textit{Hierarchische Methoden am Beispiel von Schiffe versenken},
	\newblock Bachelorarbeit Nr.25,
	\newblock Universität Stuttgart,
	\newblock 2013
	
	\bibitem[P10]{P10}
	\textsc{Pflüger}, D.:
	\newblock \textit{Spatially Adaptive Sparse Grids for High-Dimensional Problems},
	\newblock Dissertation,,
	\newblock Technische Universität München,
	\newblock 2010
	
	\bibitem[GG08]{GG08}
	\textsc{Gerstner}, T. ; \textsc{Griebel}, M.:
	\newblock \textit{Sparse Grids},
	\newblock From Encyclopedia of Quantitative Finance,
	\newblock Universität Bonn,
	\newblock 2008
	
\end{thebibliography}


%\iffalse
\end{document}
%\fi
