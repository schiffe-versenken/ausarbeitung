% This is LLNCS.DEM the demonstration file of
% the LaTeX macro package from Springer-Verlag
\documentclass[a4paper,12pt]{llncs}
%
\usepackage{makeidx}  % allows for indexgeneration
\makeindex

\usepackage[ngerman]{babel}
\usepackage[utf8]{inputenc}      % Code-Page latin 1
\usepackage[T1]{fontenc}
% Nur eine der beiden folgenden Zeilen einbinden!
% siehe Abschnitt Bilder
%\usepackage{graphicx}       % Bilder einbinden, Version fuer normales latex
\usepackage[pdftex]{graphicx}       % Bilder einbinden, Version fuer pdflatex

% mit Hyperrefs
\usepackage[pdftex, plainpages=false,hypertexnames=true,pdfnewwindow=true,backref=true,colorlinks=true,citecolor=blue,linkcolor=black,urlcolor=blue,filecolor=blue]{hyperref}% 
% weitere Packages
\usepackage{ifthen}                 % Zum Auskommentieren von Textteilen
\usepackage{amssymb}                % Mathematische Buchstaben
\usepackage{amsmath}                % Verbesserter Formelsatz
\usepackage[vlined,boxed]{algorithm2e}
\usepackage{booktabs}               % schönere Tabellen
\usepackage{color}
\usepackage{hyperref}
 \hypersetup{urlcolor=black,citecolor=black}
%\setalcapskip{1.5ex} % fuer package algorithm
\usepackage{dsfont}  
%\newtheorem{definition}{Definition}
\usepackage{doc}

% Seitenformat ===============================================================
\hoffset=-1.25truecm
\setlength{\topmargin}{0.0cm}
\setlength{\textheight}{23.0cm}
\setlength{\footskip}{1.5cm}
\setlength{\textwidth}{15.4cm}
\setlength{\evensidemargin}{1.5cm}
\setlength{\oddsidemargin}{1.5cm}
\setlength{\parskip}{1ex}
\setlength{\parindent}{0pt}
\setlength{\marginparwidth}{1.4cm}
\setlength{\marginparsep}{1mm}

\pagestyle{plain}

% Makro-Definitionen ==========================================================
% Zahlenbereiche -------------------------------------------------------------
\newcommand{\N}{{\mathbb{N}}}
\newcommand{\R}{{\mathbb{R}}}
\newcommand{\C}{{\mathbb{C}}}
\newcommand{\Z}{{\mathbb{Z}}}
\newcommand{\Q}{{\mathbb{Q}}}

% 
\def\myverzeichnis{.}

\numberwithin{equation}{section} 
% Bild -----------------------------------------------------------------------
% #1 Filename;  #2 Label;  #3 Bildunterschrift;  #4 Kurzform
\newcommand{\bild}[4]{
  \begin{figure}[htbp]
    \begin{center}
      \includegraphics{#1}
      \caption[#4]{#3}
      \label{#2}
    \end{center}
  \end{figure}
}

% Bildbreite -----------------------------------------------------------------
% #1 Filename;  #2 Breite;  #3 Label;  #4 Bildunterschrift;  #5 Kurzform
\newcommand{\bildbreite}[5]{
  \begin{figure}[htbp]
    \begin{center}
      \includegraphics[width=#2]{#1}
      \caption[#5]{#4}
      \label{#3}
    \end{center}
  \end{figure}
}

\newtheorem{satz}{Satz}


% ============================================================================
\begin{document}

% =========== Das war der Vorspann, jetzt geht's los! ========================

% ============================================================================
% =============  AB HIER DARF UND SOLL GETIPPT WERDEN ========================
% ============================================================================

\author{Viel Schreiber}
\index{Viel Schreiber}

% Das Institut wird fuer den Betreuer missbraucht ...
\institute{{\bf Betreuer:} Dipl.-Inf. Carl Coder}
\authorrunning{Viel Schreiber}
\title{Meine Seminarausarbeitung}

\maketitle

\thispagestyle{empty}

\begin{abstract}
Ein schöner Abstract. Das ist einfach die Kurzzusammenfassung.
\end{abstract}

% Einleitung -----------------------------------------------------------------
\section{Spielbeschreibung}
Lorem ipsum dolor sit amet, consetetur sadipscing elitr, sed diam nonumy eirmod tempor invidunt ut labore et dolore magna aliquyam erat, sed diam voluptua. At vero eos et accusam et justo duo dolores et ea rebum. Stet clita kasd gubergren, no sea takimata sanctus est Lorem ipsum dolor sit amet. Lorem ipsum dolor sit amet, consetetur sadipscing elitr, sed diam nonumy eirmod tempor invidunt ut labore et dolore magna aliquyam erat, sed diam voluptua. At vero eos et accusam et justo duo dolores et ea rebum. Stet clita kasd gubergren, no sea takimata sanctus est Lorem ipsum dolor sit amet.

\section{Optimale Schüsse berechnen}

\subsection{Definitionen}
\begin{definition}
Sei $C_{all}=\{1, \dots, N\}^d$ die Menge aller Zellen (cells) des Spielfeldes mit jeweils $N$ Zellen in $d$ Dimensionen.
\end{definition}

\begin{definition}
Sei $l$ eine mögliche Schiffsposition (location), welche mithilfe einer minimalen Ecke $c_{min}(l)$ und einer maximalen Ecke $c_{max}(l)$ bestimmt wird. Dann ist
\[
c(l)=
\{
c
\in
C
\mid
c_{min}(l) \leq c \leq c_{max}(l)
\}
\]
die Menge aller Zellen, welche sich innerhalb der möglichen Schiffsposition $l$ befinden.
\end{definition}

\begin{definition}
Sei 
\[
L_{all}=
\{
\{
k
\in
C_{all}
\mid
i \leq k \leq j
\}
\mid
i,j \in C_{all}
\wedge
i \leq j
\}
\] die Menge aller möglichen Schiffspositionen.
\end{definition}

\begin{definition}
Sei $m$ die Anzahl der platzierten Schiffe.
\end{definition}

\begin{definition}
$all\_distributions=\{L \subseteq L_{all} \mid |L|=m\}$ ist die Menge von allen möglichen Schiffsverteilungen.
\end{definition}

\begin{definition}
Sei $L_a \in all\_distributions$ die eigentliche und geheime Schiffsverteilung.
\end{definition}

\begin{definition}
Sei $F_t=\{(c_1, h_1), \dots , (c_t, h_t)\}$ der Zustand nach $t$ Schüssen.
\end{definition}

\begin{definition}
Sei $C_{shot,F}$ die Menge an Zellen, auf die bereits geschossen wurde.
\end{definition}

\begin{definition}
Sei $C_{left,F}=C_{all} \setminus C_{shot,F}$ die Menge an Zellen, auf die noch nicht geschossen wurde.
\end{definition}

\begin{definition}
Sei
\[
D_F=\{l \in L_a \mid c(l) \subseteq C_{shot,F}\}
\]
die Menge an Schiffen, welche durch die Schüsse auf die Zellen $C_{shot,F}$ bereits versenkt wurden.
\end{definition}

\begin{definition}
Sei $ships\_left\_count_F=m - |D_F|$.
\end{definition}

\begin{definition}
Sei $H_F(c)$ die Anzahl an Treffern von einem bereits abgefeuerten Schuss auf Zelle $c$, wobei die bereits versenkten Schiffe aus $D_F$ abgezogen wurden.
\end{definition}

\subsection{Teilungsfunktionen}

\begin{satz}
Seien $C_{pos}$ die Zellen, die belegt werden sollen. Seien $C_{neg}$ die Zellen, die nicht belegt werden sollen.

Dann ist
%Sei positive\_min(d)=min(\{c_d \mid c \in C_{pos}\})
%Sei positive\_max(d)=max(\{c_d \mid c \in C_{pos}\})
%Sei b_{min}(d)=negative\_min\_border(d)=max(\{c_d \mid c \in C_{neg} \wedge c_d < positive\_min(d)\}, 1)
%Sei b_{max}(d)=negative\_max\_border(d)=min(\{c_d \mid c \in C_{neg} \wedge c_d > positive\_max(d)\}, D)\\
\[
location\_count_F(C_{pos}, C_{neg})=TODO (Dennis?)
\]
\end{satz}

\begin{definition}
Sei $cell\_combinations_F=\mathcal{P}(C_{shot,F})$.
\end{definition}

\begin{definition}
Sei $comb \in cell\_combinations_F$. Dann ist
\[
location\_count_F(comb)=location\_count_F(comb, C_{shot,F} \setminus comb)
\]
eine Kurzform.
\end{definition}

\begin{definition}
Sei $comb \in cell\_combinations_F$. Dann ist
\[
max\_shared\_ship\_count_F(comb)=min(location\_count_F(comb), \{H_F(c) \mid c \in comb\})
\]
\end{definition}

\begin{definition}
Sei
\[
s_F \colon \begin{array}{l} 
          cell\_combinations_F \rightarrow \N_0 \\ 
          comb\mapsto s_F(comb)=k, \, 0 \leq k \leq max\_shared\_ship\_count_F(comb)
         \end{array}
\]
eine Share-Funktion.
\end{definition}

\begin{definition}
Sei
\[
ship\_count(s_F)=\sum_{comb \in cell\_combinations_F}{s_F(comb)}
\]
\end{definition}

\begin{definition}
Sei
\[
hit\_count(c, s_F)=\sum_{comb \in \{x \in cell\_combinations_F \mid c \in x\}}{s_F(comb)}
\]
\end{definition}

\begin{definition}
Sei
\[
valid\_share\_functions_F=\{s_F \mid \forall{c \in C_{shot,F}}\colon hit\_count(c, s_F)=H_F(c) \wedge ship\_count(s_F) =ships\_left\_count_F\}
\]
die Menge an Share-Funktionen, die das LGS (...) lösen.
\end{definition}

\begin{definition}
Sei
\[
distribution\_count(s_F)=
\left( \prod_{comb \in cell\_combinations_F}{{location\_count_F(comb)\choose s_F(comb)}} \right)
\]
\end{definition}

\begin{definition}
Sei
\[
distribution\_count(F)=\sum_{s_F \in valid\_share\_functions_F}{distribution\_count(s_F)}
\]
\end{definition}

\subsection{Teilungsfunktionen anwenden}

Sei jetzt der momentane Zustand $F_t$.

\begin{satz}
Sei $L \subseteq ship\_distributions$ eine mögliche Schiffsverteilung.

Dann ist
\[
P(L = L_a)=\frac{1}{distribution\_count(F_t)}
\]
die Wahrscheinlichkeit, dass $L$ die richtige Schiffsverteilung ist.
\end{satz}

\begin{proof}
\[
TODO
\]
\end{proof}

\begin{satz}
Sei $c \in C_{left,F}$ die beschossene Zelle und $h \in \N_0$.
Dann gilt:
\[
P(H_F(c)=h)=\frac{distribution\_count(F_t \cup \{(c, h)\})}{distribution\_count(F_t)}
\]
\end{satz}

\begin{proof}
\[
TODO
\]
\end{proof}

\begin{theorem}
Sei $c \in C_{left,{F_t}}$ die beschossene Zelle.
Dann gilt:
\[
\mathds{E}(H_{F_t}(c))=\sum_{h=1}^{ships\_left\_count_{F_t}} P(H_F(c)=h) * h
\]
\end{theorem}

\begin{proof}
\[
TODO
\]
\end{proof}

% Literaturverzeichnis ------------------------------------------------
\newpage
\bibliographystyle{alphadinLinkLocal}
\bibliography{literatur} 

%\iffalse
\end{document}
%\fi
