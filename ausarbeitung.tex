% !TeX document-id = {88ddee8b-8eb6-46e0-9af4-2c10d8b61c04}
% !TeX spellcheck = de-DE
% !TeX encoding = utf8
% !TeX TXS-program:compile = txs:///pdflatex/[--shell-escape]

% This is LLNCS.DEM the demonstration file of
% the LaTeX macro package from Springer-Verlag
\documentclass[a4paper,12pt]{llncs}
%
\usepackage{makeidx}  % allows for indexgeneration
\makeindex

\usepackage[ngerman]{babel}
\usepackage[utf8]{inputenc}      % Code-Page latin 1
\usepackage[T1]{fontenc}
% Nur eine der beiden folgenden Zeilen einbinden!
% siehe Abschnitt Bilder
%\usepackage{graphicx}       % Bilder einbinden, Version fuer normales latex
\usepackage[pdftex]{graphicx}       % Bilder einbinden, Version fuer pdflatex

% mit Hyperrefs
\usepackage[pdftex, plainpages=false,hypertexnames=true,pdfnewwindow=true,backref=true,colorlinks=true,citecolor=blue,linkcolor=black,urlcolor=blue,filecolor=blue]{hyperref}% 
% weitere Packages
\usepackage{ifthen}                 % Zum Auskommentieren von Textteilen
\usepackage{amssymb}                % Mathematische Buchstaben
\usepackage{amsmath}                % Verbesserter Formelsatz
\usepackage[vlined,boxed]{algorithm2e}
\usepackage{booktabs}               % schönere Tabellen
\usepackage{color}
\usepackage{algpseudocode}			% Für Pseudocode Algorithmen
\usepackage{tcolorbox}
\usepackage{hyperref}
 \hypersetup{urlcolor=black,citecolor=black}

%\setalcapskip{1.5ex} % fuer package algorithm
\usepackage{dsfont}  
%\newtheorem{definition}{Definition}
\usepackage{doc}
\usepackage{mathrsfs}
%TODO Die nächsten Zeilen sollten in der finalen Version deaktiviert werden.
%\usepackage[local,grumpy,dirty={, geändert}]{gitinfo2}
\usepackage{mathtools}
\usepackage{todonotes}

% Seitenformat ===============================================================
\hoffset=-1.25truecm
\setlength{\topmargin}{0.0cm}
\setlength{\textheight}{23.0cm}
\setlength{\footskip}{1.5cm}
\setlength{\textwidth}{15.4cm}
\setlength{\evensidemargin}{1.5cm}
\setlength{\oddsidemargin}{1.5cm}
\setlength{\parskip}{1ex}
\setlength{\parindent}{0pt}
\setlength{\marginparwidth}{1.4cm}
\setlength{\marginparsep}{1mm}

\pagestyle{plain}

% Makro-Definitionen ==========================================================

% 
\def\myverzeichnis{.}

\numberwithin{equation}{section} 
% Bild -----------------------------------------------------------------------
% #1 Filename;  #2 Label;  #3 Bildunterschrift;  #4 Kurzform
\newcommand{\bild}[4]{
  \begin{figure}[htbp]
    \begin{center}
      \includegraphics{#1}
      \caption[#4]{#3}
      \label{#2}
    \end{center}
  \end{figure}
}

% Bildbreite -----------------------------------------------------------------
% #1 Filename;  #2 Breite;  #3 Label;  #4 Bildunterschrift;  #5 Kurzform
\newcommand{\bildbreite}[5]{
  \begin{figure}[htbp]
    \begin{center}
      \includegraphics[width=#2]{#1}
      \caption[#5]{#4}
      \label{#3}
    \end{center}
  \end{figure}
}

\newtheorem{satz}{Satz}
\newtheorem{korollar}{Korollar}

\DeclareMathOperator{\hit}{hit}
\DeclareMathOperator{\shot_count}{shot\_count}
\DeclareMathOperator{\strat}{strat}
\DeclareMathOperator{\strategies}{strategies}
\DeclareMathOperator{\sunk}{sunk}


% ============================================================================
\begin{document}

% =========== Das war der Vorspann, jetzt geht's los! ========================

% ============================================================================
% =============  AB HIER DARF UND SOLL GETIPPT WERDEN ========================
% ============================================================================
\author{Viel Schreiber}
\index{Viel Schreiber}

% Das Institut wird fuer den Betreuer missbraucht ...
\institute{{\bf Betreuer:} Dipl.-Inf. Carl Coder}
\authorrunning{Viel Schreiber}
\title{Meine Seminarausarbeitung}

\maketitle

%\todo[inline]{Version \gitAbbrevHash, Branch \gitBranch, \gitCommitterIsoDate \gitDirty}

\thispagestyle{empty}

\begin{abstract}
Ein schöner Abstract. Das ist einfach die Kurzzusammenfassung.
\end{abstract}

% Einleitung -----------------------------------------------------------------
\section{Einleitung}

\subsection{Spielbeschreibung}
\nocite{WB95} % Mindestens eine Zitierung ist nötig, damit BibTex funktioniert
Diese Ausarbeitung beschäftigt sich mit der Umsetzung von Schiffe-Versenken in beliebig viele Dimensionen.
Die generelle Funktionsweise von dem normalen Schiffe-Versenken bleibt erhalten, muss jedoch um einige Dinge erweitert werden, um auch in höheren Dimensionen gut spielbar zu sein.

Zu Beginn platziert der Gegner eine beliebige Anzahl an Schiffen auf seinem Spielfeld. Es wird davon ausgegangen, dass die Schiffspositionen rein zufällig ausgewählt wurden, d.h. dass zu Spielbeginn eine Flotte aus der Menge aller möglichen Flotten ausgewählt wird. Jede mögliche Flotte hat hierbei die gleiche Wahrscheinlichkeit, ausgewählt zu werden.
Natürlich sind die Schiffspositionen dem Spieler nicht bekannt.
Die Gesamtheit aller gewählten Schiffspositionen wird auch die Flotte genannt.
Nun kann der Spieler anfangen, auf bestimmte Positionen auf dem Spielfeld, auch Zellen genannt, zu schießen.
Da Schiffe überlappen können, können mit jedem Schuss mehrere Schiffe, also Teilmengen der Flotte getroffen werden.
Nach jedem Schuss erfährt der Spieler von seinem Gegner, welche Teilmenge der Flotte er getroffenen hat.
Falls keine Schiffe getroffen wurden ist diese Menge natürlich leer.
Ein Schiff gilt als versenkt, sobald es getroffen wurde.
Sobald alle Schiffe versenkt wurden, ist das Spiel beendet.
Das Ziel des Spielers ist es, mit möglichst wenig Schüssen alle Schiffe zu versenken.

\subsection{Inhalt der Ausarbeitung}

Der erste Teil dieser Ausarbeitung beschäftigt sich mit der Bewertung einer gegebenen Strategie, d.h. mit der Berechnung der erwarteten Anzahl an Schüssen, die mit einer Strategie zum versenken der platzierten Flotte gebraucht werden.
Da diese Berechnung sehr aufwändig ist, wird ein Monte-Carlo Algorithmus zur nährweisen Berechnung beschrieben, implementiert und ausgewertet.

Zweiter Teil TODO

\section{Formalisierung des Spielprinzips}

\begin{definition}
Sei
\[
C_{all} \coloneqq \{1, \dots, N\}^d
\]
die Menge aller Zellen des Spielfeldes mit jeweils $N$ Zellen in $d$ Dimensionen.
\end{definition}

\begin{satz}
Dann gilt:
\[
|C_{all}|=N^d
\]
\end{satz}

\begin{definition}
Seien $c, c' \in C_{all}$ Zellen.
Dann ist
\[
c \leq c' \Leftrightarrow \forall i \in \{1, \dots, d\} \colon c_{i} \leq c'_{i} 
\]
\end{definition}

\begin{definition}
Sei $c_{min} \in C_{all}$ und $c_{max} \in C_{all}$ mit $c_{min} \leq c_{max}$.
\[
l=(c_{min}, c_{max})
\]
eine mögliche Schiffsposition (location), welche mithilfe einer minimalen Ecke $c_{min}$ und einer maximalen Ecke $c_{max}$ bestimmt wird.
\end{definition}

\begin{definition}
Sei 
\[
L_{all} \coloneqq
\{
(i, j) \in C_{all} \times C_{all}
\mid
i \leq j
\}
\] die Menge aller möglichen Schiffspositionen.
\end{definition}

\begin{satz}
Dann gilt:
\[
|L_{all}|=\left(\frac{(N+1) N}{2}\right)^d
\]
\end{satz}

\begin{proof}
Da das Spielfeld in jeder Dimension die Größe $N$ hat, gilt in jeder Dimension $d$ für gültige Start- und Endkoordinaten eines Schiffes $c_d$ und $c'_d$ mit $c_d \leq c'_d$ folgendes:
\begin{align}
\begin{split}
&|\{(c_d, c'_d) \in N \times N \mid c_d \leq c'_d\}|\\
&=|\{(i, j) \in N \times N \mid i \leq j\}|\\
&=\sum_{i=1}^N N - i + 1\\
&=N + \sum_{i=1}^{N-1} i\\
&=\sum_{i=1}^{N} i\\
&= \frac{(N + 1) N}{2}
\nonumber
\end{split}
\end{align}
Daher gibt es genau
\[
\left(\frac{(N+1) N}{2}\right)^d
\]
gültige Schiffspositionen, da für jede Schiffsposition für jede Dimension gültige Start- und Endkoordinaten gewählt werden müssen, also genau $d$ gültige Start- und Endkoordinaten.
\end{proof}

\begin{definition}
Sei 
\[
K \coloneqq \{1, \dots, |L_{all}|\}
\]
die Menge aller möglichen Flottengrößen.
\end{definition}

\begin{definition}
Sei $k \in K$ die Größe der Flotte.
Dann ist
\[
F_{all,k} \coloneqq\{F \in P(L_{all}) \mid |F| = k\}
\]
die Menge aller Flotten der Größe $k$.
\end{definition}

\begin{satz}
Sei $k \in K$ die Größe der Flotte.
Dann gilt:
\[
|F_{all,k}|=\binom{|L_{all}|}{k}
\]
\end{satz}

\begin{definition}
Dann ist
\[
F_{all} \coloneqq \bigcup_{k=1}^{|L_{all}|} F_{all,k}
\]
\end{definition}

\begin{satz}
Dann gilt:
\[
|F_{all}|=2^{|L_{all}|} - 1
\]
\end{satz}

\begin{proof}
\begin{align}
\begin{split}
&|F_{all}|=\sum_{k=1}^{|L_{all}|} |F_{all,k}|\\
=&\sum_{k=1}^{|L_{all}|} \binom{|L_{all}|}{k} \\
=&\left( \sum_{k=0}^{|L_{all}|} \binom{|L_{all}|}{k} \right) - 1 \\
=&2^{|L_{all}|} - 1
\end{split}
\end{align}
\qed
\end{proof}


\begin{definition}
Sei $F \in F_{all}$ und $c \in C_{all}$.
Dann ist 
\begin{align}
&\hit:F_{all} \times C_{all} \rightarrow P(L_{all}) \quad mit \nonumber\\
&\hit(F, c)\mapsto \{(c_{min}, c_{max}) \in F \mid c_{min} \leq c \leq c_{max}\} \nonumber
\end{align}
die Treffer-Funktion, welche angibt, welche Schiffe bei einem Schuss auf Zelle $c$ getroffen wurden, falls $F$ die vom Gegner platzierte Flotte ist. In anderen Worten, die Menge der Schiffe aus Flotte $F$, die die Zelle $c$ belegen.
\end{definition}

\subsection{Schuss-Strategien}

\begin{definition}
Eine Funktion der Form
\begin{align}
&\strat:P(C_{all}) \rightarrow C_{all} \nonumber
\end{align}
wird Strategiefunktion genannt. Diese wählt die nächste Zelle aus, auf die geschossen werden soll, abhängig davon, auf welche Zellen bereits geschossen wurden.
\end{definition}

\begin{definition}
Sei
\[
\strategies  \coloneqq \{ \strat:P(C_{all}) \rightarrow C_{all} \}
\]
die Menge an allen Strategiefunktionen.
\end{definition}

\begin{definition}
Sei $F\in F_{all}$ die gewählte Flotte und $S \in P(C_{all})$ die Menge an bereits beschossenen Zellen.
Dann ist:
\[
\sunk(F, S) \Leftrightarrow \bigcup_{c \in S} \hit(F, c) = F
\]
wahr, falls alle Schiffe aus der Flotte $F$ versenkt wurden.
\end{definition}


\begin{definition}
Sei $F\in F_{all}$ die gewählte Flotte und $S$ die Menge an bereits beschossenen Zellen.
Sei außerdem $\strat \in \strategies$ die verwendete Schuss-Strategie.
Dann ist
\begin{align}
&\shot_count(F, S, \strat)=
& \begin{cases} 
  	0& ,\sunk(F, S) \\
      \shot_count(F, S \cup \strat(S), \strat) + 1 & ,sonst
   \end{cases}
\nonumber
\end{align}
die Anzahl an Schüssen, die benötigt werden, um alle Schiffe der Flotte $F$ mit der Schuss-Strategiefunktion $\strat$ zu versenken, falls bereits auf die Zellen der Menge $S$ geschossen wurde.

Außerdem ist dann
\begin{align}
&\shot_count(F, \strat)=\shot_count(F, \emptyset, \strat)
\nonumber
\end{align}
die Anzahl an Schüssen, die benötigt werden, um alle Schiffe der Flotte $F$ mit der Schuss-Strategiefunktion $\strat$ zu versenken.
\end{definition}

\subsection{Zufallsvariablen}

\begin{definition}
Sei $\Omega \subseteq F_{all}$ eine Menge an Flotten.
\end{definition}

\begin{definition}
Sei $A \subseteq \Omega$ das Ereignis, dass bei einem Auswählen einer Flotte aus $\Omega$, die Flotte ebenfalls in der Menge $A$ liegt.
Dann ist
\[
P(A) = \frac{|A|}{|\Omega|}
\]
die Wahrscheinlichkeit, dass eine Flotte aus der Menge $A$ ausgewählt wird.
\end{definition}

Dann ist $(\Omega, P)$ ein endlicher Laplacescher W-Raum und P eine Gleichverteilung.

\begin{definition}
Sei $\strat \in \strategies$ die verwendete Schuss-Strategie.
Sei $\omega$ die ausgewählte Flotte aus $\Omega$.
Dann ist
\begin{align}
\mathbf{T}_{\strat,\Omega}(\omega) \coloneqq \shot_count(\omega, \strat)
\nonumber
\end{align}
eine Zufallsvariable, die die Anzahl an Schüssen zum versenken einer Flotte $\omega$ bezeichnet.
\end{definition}

\begin{satz}
Sei $\strat \in \strategies$ die verwendete Schuss-Strategie.
Dann ist
\begin{align}
E(\mathbf{T}_{\strat, \Omega})=\frac{1}{|\Omega|} \sum_{\omega \in \Omega} \shot_count(\omega, \strat)
\nonumber
\end{align}
der Erwartungswert von $\mathbf{T}_{\strat, \Omega}$.
\end{satz}

\begin{proof}
\begin{align}
\begin{split}
E(\mathbf{T}_{\strat,\Omega})&=\sum_{\omega \in \Omega} P(\omega) \shot_count(\omega, \strat)\\
&=\frac{1}{|\Omega|} \sum_{\omega \in \Omega} \shot_count(\omega, \strat)
\nonumber
\end{split}
\end{align}
\qed
\end{proof}

\begin{satz}
Sei $\strat \in \strategies$ die verwendete Schuss-Strategie.
Dann ist
\begin{align}
E(\mathbf{T}_{\strat,F_{all}})=\frac{1}{|F_{all}|} \sum_{k=1}^{|L_{all}|}\sum_{F\in F_{all,k}} \shot_count(F, \strat)
\nonumber
\end{align}
der Erwartungswert von $\mathbf{T}_{\strat,F_{all}}$.
\end{satz}

\begin{proof}
\begin{align}
\begin{split}
&E(\mathbf{T}_{\strat,F_{all}})\\
=&\frac{1}{|F_{all}|} \sum_{F\in F_{all}} \shot_count(F, \strat)\\
=&\frac{1}{|F_{all}|} \sum_{k=1}^{|L_{all}|}\sum_{F\in F_{all,k}} \shot_count(F, \strat)
\nonumber
\end{split}
\end{align}
\qed
\end{proof}

\begin{definition}
Sei $k \in K$ die Flottengröße.
Dann ist
\[
g_k=\frac{|F_{all,k}|}{|F_{all}|}=\frac{\binom{|L_{all}|}{k}}{2^{|L_{all}|} - 1}
\]
der Anteil an Flotten der Größe $k$ an der Menge aller Flotten. Dieser Faktor wird später als Gewichtungsfaktor für den Erwartungswert der Flottengröße $k$ verwendet.
\end{definition}

\begin{satz}
Sei $\strat \in \strategies$ die verwendete Schuss-Strategie.
Dann ist
\begin{align}
E(\mathbf{T}_{\strat,F_{all}})=\sum_{k=1}^{|L_{all}|} g_k E(\mathbf{T}_{\strat,F_{all,k}})
\nonumber
\end{align}
der Erwartungswert von $\mathbf{T}_{\strat,F_{all}}$.
\end{satz}

\begin{proof}
\begin{align}
\begin{split}
&E(\mathbf{T}_{\strat,F_{all}})\\
=&\frac{1}{|F_{all}|} \sum_{k=1}^{|L_{all}|}\sum_{F\in F_{all,k}} \shot_count(F, \strat)\\
=&\sum_{k=1}^{|L_{all}|} \frac{1}{|F_{all}|} \sum_{F\in F_{all,k}} \shot_count(F, \strat)\\
=&\sum_{k=1}^{|L_{all}|} \frac{1}{|F_{all}|} |F_{all,k}| \sum_{F\in F_{all,k}} \frac{1}{|F_{all,k}|} \shot_count(F, \strat)\\
=&\sum_{k=1}^{|L_{all}|} \frac{|F_{all,k}|}{|F_{all}|} E(\mathbf{T}_{\strat,F_{all,k}})\\
=&\sum_{k=1}^{|L_{all}|} g_k E(\mathbf{T}_{\strat,F_{all,k}})
\nonumber
\end{split}
\end{align}
\qed
\end{proof}

\subsection{Das Gesetz der großen Zahlen}

\begin{definition}
Sei im folgenden Abschnitt $\epsilon \in R_+$ die obere Grenze für die akzeptierte Abweichung des errechneten Erwartungswertes vom tatsächlichen Erwartungswert.

Sei im folgenden Abschnitt $\eta \in (0,1]$ die untere Grenze für die akzeptierte Wahrscheinlichkeit, dass Abweichung des errechneten Erwartungswertes kleiner als $\epsilon$ ist. 
\end{definition}

\begin{satz}
Sei $\epsilon > 0$ und $F_{sub,k} \subseteq F_{all,k}$ eine Menge an Flotten der Größe $k$.
Dann gilt nach dem schwachen Gesetz der großen Zahlen:
\begin{align}
P(|E(\mathbf{T}_{\strat,F_{sub,k}}) - E(\mathbf{T}_{\strat,F_{all,k}})| \geq \epsilon) \leq \frac{V(\mathbf{T}_{\strat,F_{sub,k}})}{|F_{sub,k}| \epsilon^2}
\nonumber
\end{align}
\end{satz}

Diese Aussage kann folgendermaßen umgeformt werden:

\begin{satz}
Sei $F_{sub,k} \subseteq F_{all,k}$ eine Menge an Flotten der Größe $k$.
Dann gilt:
\begin{align}
P(|E(\mathbf{T}_{\strat,F_{sub,k}}) - E(\mathbf{T}_{\strat,F_{all,k}})| < \epsilon) \geq 1 - \frac{V(\mathbf{T}_{\strat,F_{sub,k}})}{|F_{sub,k}| \epsilon^2}
\nonumber
\end{align}
\end{satz}

\begin{proof}
\begin{align}
\begin{split}
&P(|E(\mathbf{T}_{\strat,F_{sub,k}}) - E(\mathbf{T}_{\strat,F_{all,k}})| \geq \epsilon) \leq \frac{V(\mathbf{T}_{\strat,F_{sub,k}})}{|F_{sub,k}| \epsilon^2}\\
\Leftrightarrow &1 - P(|E(\mathbf{T}_{\strat,F_{sub,k}}) - E(\mathbf{T}_{\strat,F_{all,k}})| < \epsilon) \leq \frac{V(\mathbf{T}_{\strat,F_{sub,k}})}{|F_{sub,k}| \epsilon^2}\\
\Leftrightarrow &P(|E(\mathbf{T}_{\strat,F_{sub,k}}) - E(\mathbf{T}_{\strat,F_{all,k}})| < \epsilon) \geq 1 - \frac{V(\mathbf{T}_{\strat,F_{sub,k}})}{|F_{sub,k}| \epsilon^2}\\
\end{split}
\end{align}
\qed
\end{proof}

\begin{satz}
Damit kann dann eine Abbruchbedingung formuliert werden, welche festlegt, ab wann der mit $F_{sub,k}$ berechnete Erwartungswert für uns gut genug ist:

\begin{align}
\frac{V(\mathbf{T}_{\strat,F_{sub,k}})}{|F_{sub,k}| \epsilon^2} \geq 1 - \eta \Leftrightarrow P(|E(\mathbf{T}_{\strat,F_{sub,k}}) - E(\mathbf{T}_{\strat,F_{all,k}})| < \epsilon) \geq \eta
\nonumber
\end{align}
\end{satz}

\section{Berechnung des Erwartungswertes mithilfe eines MC-Algorithmus}

Um zu bewerten, wie gut eine Strategiefunktion $\strat$ abschneidet, kann ein Monte-Carlo Algorithmus implementiert werden, der mit einer gewissen Wahrscheinlichkeit und Genauigkeit für jede Flottengröße $k \in K$ den Wert von $E(\mathbf{T}_{\strat,F_{all,k}})$ abschätzt und am Ende alle diese Werte zum gesamtem Erwartungswert $E(\mathbf{T}_{\strat,F_{all}})$ kombiniert.

Um den Wert von $\shot_count(F, \strat)$ für eine gegebene Flotte $F$ zu berechnen, kann folgende Funktion verwendet werden:

\begin{algorithm}[H]
 function shot\_count(F, strat):\\
 $S=\emptyset$\;
 $\shot_count=0$\;
 \While{$\neg \sunk(F, S)$}{
  $next\_target=\strat(S)$\;
  $S=S \cup \hit(F, next\_target)$\;
  $\shot_count=\shot_count+1$\;
 }
  \Return $\shot_count$\;
\end{algorithm}


\subsection{Abbruchbedingung}

Dann kann eine Funktion implementiert werden, die den Wert von $E(\mathbf{T}_{\strat,F_{all,k}})$ näherungsweise berechnet. Da es sehr viele Flotten gibt, werden nur so viele Flotten aus dieser Menge ausgewertet, bis der berechnete Wert für uns genau genug ist. Dafür wird in jedem Durchlauf der Teilmenge $F_{sub,k}$ eine weitere zufällige Flotte der Größe $k$ hinzugefügt und danach überprüft, ob die vorher definierte Abbruchbedingung
\[
\frac{V(\mathbf{T}_{\strat,F_{sub,k}})}{|F_{sub,k}| \epsilon^2} \geq 1 - \eta
\]
erfüllt ist:

\begin{algorithm}[H]
 function simulate\_rounds(k, strat, $\epsilon, \eta$):\\
 $sum=0$\;
 $q\_sum=0$\;
   $E=0$\;
   $V=0$\;
    $rounds =0$\;
 \While{$true$}{
   $count += shot\_count(generate\_random\_fleet(k), \strat)$\;
   $sum += count$\; 
   $q\_sum += {count}^2$\;
   $E =sum / rounds$\;
   $V =(q\_sum / rounds) - E^2$\;
     \If {$V / (rounds * \epsilon^2) \geq 1 - \eta$}{
     break\;
   }
 }
 \Return $E$\;
\end{algorithm}

Abschliessend kann dann die Funktion implementiert werden, die den Wert von $E(\mathbf{T}_{\strat,F_{all}})$ für Flotten beliebiger Größe berechnet. Dafür müssen die Werte für jedes $k$ gewichtet aufsummiert werden. Nebenbei werden auch die Erwartungswerte für jedes $k$ gespeichert, damit diese später auch untersucht werden können.

\begin{algorithm}[H]
 function calculate\_expected\_shot\_count(strat, $\epsilon, \eta$):\\
 $expected\_shot\_count=[]$\;
 $total\_expected\_shot\_count=0$\;
 \For{$k=1,\dots, |L_{all}|$}{
 $(E,V)=simulate\_rounds(k, strat, \epsilon, \eta)$\;
 $expected\_shot\_count[k] = E$\;
 $g_k=\frac{|F_{all,k}|}{|F_{all}|}$\;
  $total\_expected\_shot\_count+=g_k * E$\;
 }
\end{algorithm}

\section{Strategien}

Im folgenden Abschnitt sollen einige Schussstrategien näher betrachtet werden. Dabei wird zuerst die aus der Bachelorarbeit bekannte optimale Strategie in unserer Notation noch einmal dargestellt. Anschließend werden noch andere Strategien vorgestellt.

Sei für alle folgenden Strategien $F\in F_{all}$ die aktuell vorliegende unbekannte Flotte.

\subsection{Maximum Strategie}

Diese in der Bachelorarbeit vorgestellte Strategie berechnet zunächst eine Anfangsverteilung. Dabei wird für jeden Punkt im betrachteten $d$-dimensionalen Raum festgestellt wie viele Schiffe diesen Punkt enthalten. Anschließend wird dem Punkt diese Anzahl zugewiesen. Von diesen Punkten wird dann das Maximum bezüglich dieser Anzahl bestimmt. Das ist dann die Zelle auf die als nächstes geschossen wird. Der Strategie wird also der aktuelle Zustand, auf welche Zellen bereits geschossen wurde, in $S$ übergeben und diese bestimmt daraus das nächste Ziel für einen Schuss.


\begin{tcolorbox}
	\begin{algorithmic}
		\Function{max\_strat}{$S$}{$:cell$}
		\State $cell = null$;
		\State $max = 0$;
		\State $temp = 0$;
		\State\ForEach{$c\in C_{all}$}{
		\State $temp=\hit(F,c)$;
		\State\If{$temp>max$}{
		\State $max=temp$;
		\State $cell = c$;
		}
		}
		\State \Return $cell$;
		\EndFunction
	\end{algorithmic}
\end{tcolorbox}

Wie man sieht bewegen wir uns hier bereits in einer Komplexität von $O(\left|C_{all}\right|\cdot O(hit))$. %O(hit) aktuell noch unbekannt, daher unformelle Notation.

\subsection{Monte-Carlo Strategie}
\subsection{Quasi-Monte-Carlo Strategie}


% Literaturverzeichnis ------------------------------------------------
\newpage
\bibliographystyle{alphadinLinkLocal}
\bibliography{literatur} 

%\iffalse
\end{document}
%\fi
